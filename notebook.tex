
% Default to the notebook output style

    


% Inherit from the specified cell style.




    
\documentclass[11pt]{article}

    
    
    \usepackage[T1]{fontenc}
    % Nicer default font (+ math font) than Computer Modern for most use cases
    \usepackage{mathpazo}

    % Basic figure setup, for now with no caption control since it's done
    % automatically by Pandoc (which extracts ![](path) syntax from Markdown).
    \usepackage{graphicx}
    % We will generate all images so they have a width \maxwidth. This means
    % that they will get their normal width if they fit onto the page, but
    % are scaled down if they would overflow the margins.
    \makeatletter
    \def\maxwidth{\ifdim\Gin@nat@width>\linewidth\linewidth
    \else\Gin@nat@width\fi}
    \makeatother
    \let\Oldincludegraphics\includegraphics
    % Set max figure width to be 80% of text width, for now hardcoded.
    \renewcommand{\includegraphics}[1]{\Oldincludegraphics[width=.8\maxwidth]{#1}}
    % Ensure that by default, figures have no caption (until we provide a
    % proper Figure object with a Caption API and a way to capture that
    % in the conversion process - todo).
    \usepackage{caption}
    \DeclareCaptionLabelFormat{nolabel}{}
    \captionsetup{labelformat=nolabel}

    \usepackage{adjustbox} % Used to constrain images to a maximum size 
    \usepackage{xcolor} % Allow colors to be defined
    \usepackage{enumerate} % Needed for markdown enumerations to work
    \usepackage{geometry} % Used to adjust the document margins
    \usepackage{amsmath} % Equations
    \usepackage{amssymb} % Equations
    \usepackage{textcomp} % defines textquotesingle
    % Hack from http://tex.stackexchange.com/a/47451/13684:
    \AtBeginDocument{%
        \def\PYZsq{\textquotesingle}% Upright quotes in Pygmentized code
    }
    \usepackage{upquote} % Upright quotes for verbatim code
    \usepackage{eurosym} % defines \euro
    \usepackage[mathletters]{ucs} % Extended unicode (utf-8) support
    \usepackage[utf8x]{inputenc} % Allow utf-8 characters in the tex document
    \usepackage{fancyvrb} % verbatim replacement that allows latex
    \usepackage{grffile} % extends the file name processing of package graphics 
                         % to support a larger range 
    % The hyperref package gives us a pdf with properly built
    % internal navigation ('pdf bookmarks' for the table of contents,
    % internal cross-reference links, web links for URLs, etc.)
    \usepackage{hyperref}
    \usepackage{longtable} % longtable support required by pandoc >1.10
    \usepackage{booktabs}  % table support for pandoc > 1.12.2
    \usepackage[inline]{enumitem} % IRkernel/repr support (it uses the enumerate* environment)
    \usepackage[normalem]{ulem} % ulem is needed to support strikethroughs (\sout)
                                % normalem makes italics be italics, not underlines
    

    
    
    % Colors for the hyperref package
    \definecolor{urlcolor}{rgb}{0,.145,.698}
    \definecolor{linkcolor}{rgb}{.71,0.21,0.01}
    \definecolor{citecolor}{rgb}{.12,.54,.11}

    % ANSI colors
    \definecolor{ansi-black}{HTML}{3E424D}
    \definecolor{ansi-black-intense}{HTML}{282C36}
    \definecolor{ansi-red}{HTML}{E75C58}
    \definecolor{ansi-red-intense}{HTML}{B22B31}
    \definecolor{ansi-green}{HTML}{00A250}
    \definecolor{ansi-green-intense}{HTML}{007427}
    \definecolor{ansi-yellow}{HTML}{DDB62B}
    \definecolor{ansi-yellow-intense}{HTML}{B27D12}
    \definecolor{ansi-blue}{HTML}{208FFB}
    \definecolor{ansi-blue-intense}{HTML}{0065CA}
    \definecolor{ansi-magenta}{HTML}{D160C4}
    \definecolor{ansi-magenta-intense}{HTML}{A03196}
    \definecolor{ansi-cyan}{HTML}{60C6C8}
    \definecolor{ansi-cyan-intense}{HTML}{258F8F}
    \definecolor{ansi-white}{HTML}{C5C1B4}
    \definecolor{ansi-white-intense}{HTML}{A1A6B2}

    % commands and environments needed by pandoc snippets
    % extracted from the output of `pandoc -s`
    \providecommand{\tightlist}{%
      \setlength{\itemsep}{0pt}\setlength{\parskip}{0pt}}
    \DefineVerbatimEnvironment{Highlighting}{Verbatim}{commandchars=\\\{\}}
    % Add ',fontsize=\small' for more characters per line
    \newenvironment{Shaded}{}{}
    \newcommand{\KeywordTok}[1]{\textcolor[rgb]{0.00,0.44,0.13}{\textbf{{#1}}}}
    \newcommand{\DataTypeTok}[1]{\textcolor[rgb]{0.56,0.13,0.00}{{#1}}}
    \newcommand{\DecValTok}[1]{\textcolor[rgb]{0.25,0.63,0.44}{{#1}}}
    \newcommand{\BaseNTok}[1]{\textcolor[rgb]{0.25,0.63,0.44}{{#1}}}
    \newcommand{\FloatTok}[1]{\textcolor[rgb]{0.25,0.63,0.44}{{#1}}}
    \newcommand{\CharTok}[1]{\textcolor[rgb]{0.25,0.44,0.63}{{#1}}}
    \newcommand{\StringTok}[1]{\textcolor[rgb]{0.25,0.44,0.63}{{#1}}}
    \newcommand{\CommentTok}[1]{\textcolor[rgb]{0.38,0.63,0.69}{\textit{{#1}}}}
    \newcommand{\OtherTok}[1]{\textcolor[rgb]{0.00,0.44,0.13}{{#1}}}
    \newcommand{\AlertTok}[1]{\textcolor[rgb]{1.00,0.00,0.00}{\textbf{{#1}}}}
    \newcommand{\FunctionTok}[1]{\textcolor[rgb]{0.02,0.16,0.49}{{#1}}}
    \newcommand{\RegionMarkerTok}[1]{{#1}}
    \newcommand{\ErrorTok}[1]{\textcolor[rgb]{1.00,0.00,0.00}{\textbf{{#1}}}}
    \newcommand{\NormalTok}[1]{{#1}}
    
    % Additional commands for more recent versions of Pandoc
    \newcommand{\ConstantTok}[1]{\textcolor[rgb]{0.53,0.00,0.00}{{#1}}}
    \newcommand{\SpecialCharTok}[1]{\textcolor[rgb]{0.25,0.44,0.63}{{#1}}}
    \newcommand{\VerbatimStringTok}[1]{\textcolor[rgb]{0.25,0.44,0.63}{{#1}}}
    \newcommand{\SpecialStringTok}[1]{\textcolor[rgb]{0.73,0.40,0.53}{{#1}}}
    \newcommand{\ImportTok}[1]{{#1}}
    \newcommand{\DocumentationTok}[1]{\textcolor[rgb]{0.73,0.13,0.13}{\textit{{#1}}}}
    \newcommand{\AnnotationTok}[1]{\textcolor[rgb]{0.38,0.63,0.69}{\textbf{\textit{{#1}}}}}
    \newcommand{\CommentVarTok}[1]{\textcolor[rgb]{0.38,0.63,0.69}{\textbf{\textit{{#1}}}}}
    \newcommand{\VariableTok}[1]{\textcolor[rgb]{0.10,0.09,0.49}{{#1}}}
    \newcommand{\ControlFlowTok}[1]{\textcolor[rgb]{0.00,0.44,0.13}{\textbf{{#1}}}}
    \newcommand{\OperatorTok}[1]{\textcolor[rgb]{0.40,0.40,0.40}{{#1}}}
    \newcommand{\BuiltInTok}[1]{{#1}}
    \newcommand{\ExtensionTok}[1]{{#1}}
    \newcommand{\PreprocessorTok}[1]{\textcolor[rgb]{0.74,0.48,0.00}{{#1}}}
    \newcommand{\AttributeTok}[1]{\textcolor[rgb]{0.49,0.56,0.16}{{#1}}}
    \newcommand{\InformationTok}[1]{\textcolor[rgb]{0.38,0.63,0.69}{\textbf{\textit{{#1}}}}}
    \newcommand{\WarningTok}[1]{\textcolor[rgb]{0.38,0.63,0.69}{\textbf{\textit{{#1}}}}}
    
    
    % Define a nice break command that doesn't care if a line doesn't already
    % exist.
    \def\br{\hspace*{\fill} \\* }
    % Math Jax compatability definitions
    \def\gt{>}
    \def\lt{<}
    % Document parameters
    \title{StackPro\_Accessment\_Solution\_Yan}
    
    
    

    % Pygments definitions
    
\makeatletter
\def\PY@reset{\let\PY@it=\relax \let\PY@bf=\relax%
    \let\PY@ul=\relax \let\PY@tc=\relax%
    \let\PY@bc=\relax \let\PY@ff=\relax}
\def\PY@tok#1{\csname PY@tok@#1\endcsname}
\def\PY@toks#1+{\ifx\relax#1\empty\else%
    \PY@tok{#1}\expandafter\PY@toks\fi}
\def\PY@do#1{\PY@bc{\PY@tc{\PY@ul{%
    \PY@it{\PY@bf{\PY@ff{#1}}}}}}}
\def\PY#1#2{\PY@reset\PY@toks#1+\relax+\PY@do{#2}}

\expandafter\def\csname PY@tok@gd\endcsname{\def\PY@tc##1{\textcolor[rgb]{0.63,0.00,0.00}{##1}}}
\expandafter\def\csname PY@tok@gu\endcsname{\let\PY@bf=\textbf\def\PY@tc##1{\textcolor[rgb]{0.50,0.00,0.50}{##1}}}
\expandafter\def\csname PY@tok@gt\endcsname{\def\PY@tc##1{\textcolor[rgb]{0.00,0.27,0.87}{##1}}}
\expandafter\def\csname PY@tok@gs\endcsname{\let\PY@bf=\textbf}
\expandafter\def\csname PY@tok@gr\endcsname{\def\PY@tc##1{\textcolor[rgb]{1.00,0.00,0.00}{##1}}}
\expandafter\def\csname PY@tok@cm\endcsname{\let\PY@it=\textit\def\PY@tc##1{\textcolor[rgb]{0.25,0.50,0.50}{##1}}}
\expandafter\def\csname PY@tok@vg\endcsname{\def\PY@tc##1{\textcolor[rgb]{0.10,0.09,0.49}{##1}}}
\expandafter\def\csname PY@tok@vi\endcsname{\def\PY@tc##1{\textcolor[rgb]{0.10,0.09,0.49}{##1}}}
\expandafter\def\csname PY@tok@vm\endcsname{\def\PY@tc##1{\textcolor[rgb]{0.10,0.09,0.49}{##1}}}
\expandafter\def\csname PY@tok@mh\endcsname{\def\PY@tc##1{\textcolor[rgb]{0.40,0.40,0.40}{##1}}}
\expandafter\def\csname PY@tok@cs\endcsname{\let\PY@it=\textit\def\PY@tc##1{\textcolor[rgb]{0.25,0.50,0.50}{##1}}}
\expandafter\def\csname PY@tok@ge\endcsname{\let\PY@it=\textit}
\expandafter\def\csname PY@tok@vc\endcsname{\def\PY@tc##1{\textcolor[rgb]{0.10,0.09,0.49}{##1}}}
\expandafter\def\csname PY@tok@il\endcsname{\def\PY@tc##1{\textcolor[rgb]{0.40,0.40,0.40}{##1}}}
\expandafter\def\csname PY@tok@go\endcsname{\def\PY@tc##1{\textcolor[rgb]{0.53,0.53,0.53}{##1}}}
\expandafter\def\csname PY@tok@cp\endcsname{\def\PY@tc##1{\textcolor[rgb]{0.74,0.48,0.00}{##1}}}
\expandafter\def\csname PY@tok@gi\endcsname{\def\PY@tc##1{\textcolor[rgb]{0.00,0.63,0.00}{##1}}}
\expandafter\def\csname PY@tok@gh\endcsname{\let\PY@bf=\textbf\def\PY@tc##1{\textcolor[rgb]{0.00,0.00,0.50}{##1}}}
\expandafter\def\csname PY@tok@ni\endcsname{\let\PY@bf=\textbf\def\PY@tc##1{\textcolor[rgb]{0.60,0.60,0.60}{##1}}}
\expandafter\def\csname PY@tok@nl\endcsname{\def\PY@tc##1{\textcolor[rgb]{0.63,0.63,0.00}{##1}}}
\expandafter\def\csname PY@tok@nn\endcsname{\let\PY@bf=\textbf\def\PY@tc##1{\textcolor[rgb]{0.00,0.00,1.00}{##1}}}
\expandafter\def\csname PY@tok@no\endcsname{\def\PY@tc##1{\textcolor[rgb]{0.53,0.00,0.00}{##1}}}
\expandafter\def\csname PY@tok@na\endcsname{\def\PY@tc##1{\textcolor[rgb]{0.49,0.56,0.16}{##1}}}
\expandafter\def\csname PY@tok@nb\endcsname{\def\PY@tc##1{\textcolor[rgb]{0.00,0.50,0.00}{##1}}}
\expandafter\def\csname PY@tok@nc\endcsname{\let\PY@bf=\textbf\def\PY@tc##1{\textcolor[rgb]{0.00,0.00,1.00}{##1}}}
\expandafter\def\csname PY@tok@nd\endcsname{\def\PY@tc##1{\textcolor[rgb]{0.67,0.13,1.00}{##1}}}
\expandafter\def\csname PY@tok@ne\endcsname{\let\PY@bf=\textbf\def\PY@tc##1{\textcolor[rgb]{0.82,0.25,0.23}{##1}}}
\expandafter\def\csname PY@tok@nf\endcsname{\def\PY@tc##1{\textcolor[rgb]{0.00,0.00,1.00}{##1}}}
\expandafter\def\csname PY@tok@si\endcsname{\let\PY@bf=\textbf\def\PY@tc##1{\textcolor[rgb]{0.73,0.40,0.53}{##1}}}
\expandafter\def\csname PY@tok@s2\endcsname{\def\PY@tc##1{\textcolor[rgb]{0.73,0.13,0.13}{##1}}}
\expandafter\def\csname PY@tok@nt\endcsname{\let\PY@bf=\textbf\def\PY@tc##1{\textcolor[rgb]{0.00,0.50,0.00}{##1}}}
\expandafter\def\csname PY@tok@nv\endcsname{\def\PY@tc##1{\textcolor[rgb]{0.10,0.09,0.49}{##1}}}
\expandafter\def\csname PY@tok@s1\endcsname{\def\PY@tc##1{\textcolor[rgb]{0.73,0.13,0.13}{##1}}}
\expandafter\def\csname PY@tok@dl\endcsname{\def\PY@tc##1{\textcolor[rgb]{0.73,0.13,0.13}{##1}}}
\expandafter\def\csname PY@tok@ch\endcsname{\let\PY@it=\textit\def\PY@tc##1{\textcolor[rgb]{0.25,0.50,0.50}{##1}}}
\expandafter\def\csname PY@tok@m\endcsname{\def\PY@tc##1{\textcolor[rgb]{0.40,0.40,0.40}{##1}}}
\expandafter\def\csname PY@tok@gp\endcsname{\let\PY@bf=\textbf\def\PY@tc##1{\textcolor[rgb]{0.00,0.00,0.50}{##1}}}
\expandafter\def\csname PY@tok@sh\endcsname{\def\PY@tc##1{\textcolor[rgb]{0.73,0.13,0.13}{##1}}}
\expandafter\def\csname PY@tok@ow\endcsname{\let\PY@bf=\textbf\def\PY@tc##1{\textcolor[rgb]{0.67,0.13,1.00}{##1}}}
\expandafter\def\csname PY@tok@sx\endcsname{\def\PY@tc##1{\textcolor[rgb]{0.00,0.50,0.00}{##1}}}
\expandafter\def\csname PY@tok@bp\endcsname{\def\PY@tc##1{\textcolor[rgb]{0.00,0.50,0.00}{##1}}}
\expandafter\def\csname PY@tok@c1\endcsname{\let\PY@it=\textit\def\PY@tc##1{\textcolor[rgb]{0.25,0.50,0.50}{##1}}}
\expandafter\def\csname PY@tok@fm\endcsname{\def\PY@tc##1{\textcolor[rgb]{0.00,0.00,1.00}{##1}}}
\expandafter\def\csname PY@tok@o\endcsname{\def\PY@tc##1{\textcolor[rgb]{0.40,0.40,0.40}{##1}}}
\expandafter\def\csname PY@tok@kc\endcsname{\let\PY@bf=\textbf\def\PY@tc##1{\textcolor[rgb]{0.00,0.50,0.00}{##1}}}
\expandafter\def\csname PY@tok@c\endcsname{\let\PY@it=\textit\def\PY@tc##1{\textcolor[rgb]{0.25,0.50,0.50}{##1}}}
\expandafter\def\csname PY@tok@mf\endcsname{\def\PY@tc##1{\textcolor[rgb]{0.40,0.40,0.40}{##1}}}
\expandafter\def\csname PY@tok@err\endcsname{\def\PY@bc##1{\setlength{\fboxsep}{0pt}\fcolorbox[rgb]{1.00,0.00,0.00}{1,1,1}{\strut ##1}}}
\expandafter\def\csname PY@tok@mb\endcsname{\def\PY@tc##1{\textcolor[rgb]{0.40,0.40,0.40}{##1}}}
\expandafter\def\csname PY@tok@ss\endcsname{\def\PY@tc##1{\textcolor[rgb]{0.10,0.09,0.49}{##1}}}
\expandafter\def\csname PY@tok@sr\endcsname{\def\PY@tc##1{\textcolor[rgb]{0.73,0.40,0.53}{##1}}}
\expandafter\def\csname PY@tok@mo\endcsname{\def\PY@tc##1{\textcolor[rgb]{0.40,0.40,0.40}{##1}}}
\expandafter\def\csname PY@tok@kd\endcsname{\let\PY@bf=\textbf\def\PY@tc##1{\textcolor[rgb]{0.00,0.50,0.00}{##1}}}
\expandafter\def\csname PY@tok@mi\endcsname{\def\PY@tc##1{\textcolor[rgb]{0.40,0.40,0.40}{##1}}}
\expandafter\def\csname PY@tok@kn\endcsname{\let\PY@bf=\textbf\def\PY@tc##1{\textcolor[rgb]{0.00,0.50,0.00}{##1}}}
\expandafter\def\csname PY@tok@cpf\endcsname{\let\PY@it=\textit\def\PY@tc##1{\textcolor[rgb]{0.25,0.50,0.50}{##1}}}
\expandafter\def\csname PY@tok@kr\endcsname{\let\PY@bf=\textbf\def\PY@tc##1{\textcolor[rgb]{0.00,0.50,0.00}{##1}}}
\expandafter\def\csname PY@tok@s\endcsname{\def\PY@tc##1{\textcolor[rgb]{0.73,0.13,0.13}{##1}}}
\expandafter\def\csname PY@tok@kp\endcsname{\def\PY@tc##1{\textcolor[rgb]{0.00,0.50,0.00}{##1}}}
\expandafter\def\csname PY@tok@w\endcsname{\def\PY@tc##1{\textcolor[rgb]{0.73,0.73,0.73}{##1}}}
\expandafter\def\csname PY@tok@kt\endcsname{\def\PY@tc##1{\textcolor[rgb]{0.69,0.00,0.25}{##1}}}
\expandafter\def\csname PY@tok@sc\endcsname{\def\PY@tc##1{\textcolor[rgb]{0.73,0.13,0.13}{##1}}}
\expandafter\def\csname PY@tok@sb\endcsname{\def\PY@tc##1{\textcolor[rgb]{0.73,0.13,0.13}{##1}}}
\expandafter\def\csname PY@tok@sa\endcsname{\def\PY@tc##1{\textcolor[rgb]{0.73,0.13,0.13}{##1}}}
\expandafter\def\csname PY@tok@k\endcsname{\let\PY@bf=\textbf\def\PY@tc##1{\textcolor[rgb]{0.00,0.50,0.00}{##1}}}
\expandafter\def\csname PY@tok@se\endcsname{\let\PY@bf=\textbf\def\PY@tc##1{\textcolor[rgb]{0.73,0.40,0.13}{##1}}}
\expandafter\def\csname PY@tok@sd\endcsname{\let\PY@it=\textit\def\PY@tc##1{\textcolor[rgb]{0.73,0.13,0.13}{##1}}}

\def\PYZbs{\char`\\}
\def\PYZus{\char`\_}
\def\PYZob{\char`\{}
\def\PYZcb{\char`\}}
\def\PYZca{\char`\^}
\def\PYZam{\char`\&}
\def\PYZlt{\char`\<}
\def\PYZgt{\char`\>}
\def\PYZsh{\char`\#}
\def\PYZpc{\char`\%}
\def\PYZdl{\char`\$}
\def\PYZhy{\char`\-}
\def\PYZsq{\char`\'}
\def\PYZdq{\char`\"}
\def\PYZti{\char`\~}
% for compatibility with earlier versions
\def\PYZat{@}
\def\PYZlb{[}
\def\PYZrb{]}
\makeatother


    % Exact colors from NB
    \definecolor{incolor}{rgb}{0.0, 0.0, 0.5}
    \definecolor{outcolor}{rgb}{0.545, 0.0, 0.0}



    
    % Prevent overflowing lines due to hard-to-break entities
    \sloppy 
    % Setup hyperref package
    \hypersetup{
      breaklinks=true,  % so long urls are correctly broken across lines
      colorlinks=true,
      urlcolor=urlcolor,
      linkcolor=linkcolor,
      citecolor=citecolor,
      }
    % Slightly bigger margins than the latex defaults
    
    \geometry{verbose,tmargin=1in,bmargin=1in,lmargin=1in,rmargin=1in}
    
    

    \begin{document}
    
    
    \maketitle
    
    

    
    \section{StackPro Coding Test}\label{stackpro-coding-test}

Yan Xu {[}04/15/2018{]} \#\# Data Description A mega-online shopping
mall provides different baby brands to customers. In order to optimize
online advertisements throughout websites, they want to predict the
view-to-click probability given the information below. The data consists
of: 1. Action: the ``interaction'' that users completed when online ad
was exposed. There are two types of actions in this dataset, View (a.k.a
Impression) or Click. (Reference:
https://en.wikipedia.org/wiki/Impression\_(online\_media)) 2. ID: unique
user IDs 3. Action Time: the time when action (View/Click) happened, in
microseconds format 4. Website: the website where the ad was published
5. Banner Size: the banner size of ad 6. Brand: the brand name that the
ad was promoting 7. Colour: the colour of product showcased in the ad 8.
Interaction Time: user's interaction time with each ad (sec)

\subsection{Task}\label{task}

\subsubsection{Goal: Predict the Likelihood of
Click}\label{goal-predict-the-likelihood-of-click}

\subsubsection{What we would like to see in your Jupyter
Notebook:}\label{what-we-would-like-to-see-in-your-jupyter-notebook}

\begin{enumerate}
\def\labelenumi{\arabic{enumi}.}
\tightlist
\item
  EDA Process (univariate/multivariate analysis, filling missing data
  etc.)
\item
  Feature Selection (select existing features and explanation)
\item
  Feature Engineering (create new features and explanation)
\item
  Select and Create Model
\item
  Hyper parameter Tuning if you feel it's needed
\item
  Model Validation process ...
\end{enumerate}

    \section{Step 1: Data Analysis: Understanding your
data}\label{step-1-data-analysis-understanding-your-data}

    \begin{Verbatim}[commandchars=\\\{\}]
{\color{incolor}In [{\color{incolor}265}]:} \PY{c+c1}{\PYZsh{}import modules}
          \PY{k+kn}{import} \PY{n+nn}{numpy} \PY{k+kn}{as} \PY{n+nn}{np} \PY{c+c1}{\PYZsh{} linear algebra}
          \PY{k+kn}{import} \PY{n+nn}{pandas} \PY{k+kn}{as} \PY{n+nn}{pd} \PY{c+c1}{\PYZsh{} data processing, CSV file I/O (e.g. pd.read\PYZus{}csv)}
          \PY{k+kn}{import} \PY{n+nn}{seaborn} \PY{k+kn}{as} \PY{n+nn}{sns}
          \PY{k+kn}{import} \PY{n+nn}{matplotlib.pyplot} \PY{k+kn}{as} \PY{n+nn}{plt}
          \PY{o}{\PYZpc{}}\PY{k}{matplotlib} inline
          
          
          \PY{k+kn}{from} \PY{n+nn}{xgboost} \PY{k+kn}{import} \PY{n}{XGBRegressor}
          
          \PY{k+kn}{from} \PY{n+nn}{sklearn.ensemble} \PY{k+kn}{import} \PY{n}{RandomForestClassifier} \PY{c+c1}{\PYZsh{} import the random forest model}
          \PY{k+kn}{from} \PY{n+nn}{sklearn} \PY{k+kn}{import}  \PY{n}{preprocessing} \PY{c+c1}{\PYZsh{} used for label encoding and imputing NaNs}
          \PY{k+kn}{from} \PY{n+nn}{sklearn.feature\PYZus{}selection} \PY{k+kn}{import} \PY{n}{RFECV}
          \PY{k+kn}{from} \PY{n+nn}{sklearn.preprocessing} \PY{k+kn}{import} \PY{n}{StandardScaler}\PY{p}{,} \PY{n}{RobustScaler}
          \PY{k+kn}{from} \PY{n+nn}{sklearn.metrics} \PY{k+kn}{import} \PY{n}{make\PYZus{}scorer} 
          \PY{k+kn}{from} \PY{n+nn}{sklearn.ensemble} \PY{k+kn}{import} \PY{n}{RandomForestRegressor}\PY{p}{,} \PY{n}{GradientBoostingRegressor}
          \PY{k+kn}{from} \PY{n+nn}{sklearn.linear\PYZus{}model} \PY{k+kn}{import} \PY{n}{Lasso}\PY{p}{,} \PY{n}{ElasticNet}\PY{p}{,} \PY{n}{Ridge}
          \PY{k+kn}{from} \PY{n+nn}{sklearn.kernel\PYZus{}ridge} \PY{k+kn}{import} \PY{n}{KernelRidge}
          \PY{k+kn}{from} \PY{n+nn}{sklearn.model\PYZus{}selection} \PY{k+kn}{import} \PY{n}{GridSearchCV}\PY{p}{,} \PY{n}{RandomizedSearchCV}\PY{p}{,} \PY{n}{cross\PYZus{}val\PYZus{}score}\PY{p}{,} \PY{n}{train\PYZus{}test\PYZus{}split}
          \PY{k+kn}{from} \PY{n+nn}{sklearn.feature\PYZus{}selection} \PY{k+kn}{import} \PY{n}{RFECV}
          \PY{c+c1}{\PYZsh{}from sklearn.preprocessing import StandardScaler, RobustScaler}
          \PY{k+kn}{from} \PY{n+nn}{sklearn.metrics} \PY{k+kn}{import} \PY{n}{make\PYZus{}scorer} 
          \PY{k+kn}{from} \PY{n+nn}{sklearn.base} \PY{k+kn}{import} \PY{n}{BaseEstimator}\PY{p}{,} \PY{n}{RegressorMixin}
          
          \PY{c+c1}{\PYZsh{} neural networks}
          \PY{k+kn}{import} \PY{n+nn}{keras}
          \PY{k+kn}{from} \PY{n+nn}{keras.models} \PY{k+kn}{import} \PY{n}{Sequential}
          \PY{k+kn}{from} \PY{n+nn}{keras.layers} \PY{k+kn}{import} \PY{n}{Dense}
          \PY{k+kn}{from} \PY{n+nn}{keras} \PY{k+kn}{import} \PY{n}{regularizers}
          
          \PY{k+kn}{from} \PY{n+nn}{sklearn.tree} \PY{k+kn}{import} \PY{n}{export\PYZus{}graphviz}
          \PY{c+c1}{\PYZsh{} import pydotplus}
          \PY{k+kn}{import} \PY{n+nn}{six}
\end{Verbatim}


    \subsection{Load in dataset}\label{load-in-dataset}

    \begin{Verbatim}[commandchars=\\\{\}]
{\color{incolor}In [{\color{incolor}266}]:} \PY{o}{\PYZpc{}}\PY{k}{ls}
\end{Verbatim}


    \begin{Verbatim}[commandchars=\\\{\}]
\textcolor{ansi-magenta-intense}{\textbf{FeatImportance.png}}
\textcolor{ansi-blue-intense}{\textbf{\_\_MACOSX}}/
StackPro\_Accessment\_Solution\_Yan.ipynb
StackPros\_Assessment\_DataScientist\_Senior.csv
StackPros\_Assessment\_DataScientist\_Senior.pdf
Untitled.ipynb
\textcolor{ansi-blue-intense}{\textbf{xgboost}}/
xgboost\_Click\_Probability.csv

    \end{Verbatim}

    \begin{Verbatim}[commandchars=\\\{\}]
{\color{incolor}In [{\color{incolor}267}]:} \PY{n}{raw\PYZus{}data} \PY{o}{=} \PY{n}{pd}\PY{o}{.}\PY{n}{read\PYZus{}csv}\PY{p}{(}\PY{l+s+s1}{\PYZsq{}}\PY{l+s+s1}{StackPros\PYZus{}Assessment\PYZus{}DataScientist\PYZus{}Senior.csv}\PY{l+s+s1}{\PYZsq{}}\PY{p}{)}
          \PY{n}{raw\PYZus{}data}\PY{o}{.}\PY{n}{head}\PY{p}{(}\PY{p}{)}
\end{Verbatim}


\begin{Verbatim}[commandchars=\\\{\}]
{\color{outcolor}Out[{\color{outcolor}267}]:}    Unnamed: 0                            ID        ActionTime Action  \textbackslash{}
          0           1  AMsySZbNg8SBwPtj7IGHnip\_8aY0  1508761859495365   View   
          1           2  AMsySZZcUFNGOJs-lfC9j-ZDlV7Z  1502847469736117   View   
          2           3  AMsySZYoK8\_bD0hGv4zu0iRn1TFo  1514169431528120   View   
          3           4  AMsySZahxpwLOWnMJj6RGATgwEOc  1515527952591896   View   
          4           5  AMsySZbqgfMBDtAqZz1jVRaOmX00  1507514809374045   View   
          
                   Website BannerSize     Brand  colour  InteractionTime  
          0       Facebook    300x600  Carter's    Pink         0.171601  
          1       Facebook    300x250  Burberry     Red         4.303945  
          2       Facebook     728x90    Disney  Yellow         0.868106  
          3       Facebook     320x50  Carter's  Yellow         0.802647  
          4  instagram.com    300x250       Gap    Gold         0.627148  
\end{Verbatim}
            
    \begin{Verbatim}[commandchars=\\\{\}]
{\color{incolor}In [{\color{incolor}268}]:} \PY{c+c1}{\PYZsh{}change the a column name \PYZsq{}Unnamed:0\PYZsq{} to \PYZsq{}Index\PYZsq{}}
          \PY{c+c1}{\PYZsh{}raw\PYZus{}data.rename(columns = \PYZob{}\PYZsq{}Unnamed: 0\PYZsq{}: \PYZsq{}Index\PYZsq{}\PYZcb{},inplace =True)}
          \PY{c+c1}{\PYZsh{} Drop Unnamed:0 column because it does not contribute to building a model}
          \PY{n}{raw\PYZus{}data}\PY{o}{.}\PY{n}{drop}\PY{p}{(}\PY{p}{[}\PY{l+s+s1}{\PYZsq{}}\PY{l+s+s1}{Unnamed: 0}\PY{l+s+s1}{\PYZsq{}}\PY{p}{]}\PY{p}{,} \PY{l+m+mi}{1}\PY{p}{,} \PY{n}{inplace}\PY{o}{=}\PY{n+nb+bp}{True}\PY{p}{)}
          \PY{n}{raw\PYZus{}data}\PY{o}{.}\PY{n}{dtypes}
\end{Verbatim}


\begin{Verbatim}[commandchars=\\\{\}]
{\color{outcolor}Out[{\color{outcolor}268}]:} ID                  object
          ActionTime           int64
          Action              object
          Website             object
          BannerSize          object
          Brand               object
          colour              object
          InteractionTime    float64
          dtype: object
\end{Verbatim}
            
    By looking at the data type of each column, we know that this dataset
mainly contains catagrical features(2 numerical features and 6
catagrical features).

    \begin{Verbatim}[commandchars=\\\{\}]
{\color{incolor}In [{\color{incolor}269}]:} \PY{c+c1}{\PYZsh{} the size of the dataset }
          \PY{k}{print} \PY{p}{(}\PY{l+s+s1}{\PYZsq{}}\PY{l+s+s1}{the Shape of The Dataset}\PY{l+s+s1}{\PYZsq{}}\PY{p}{,}\PY{n}{raw\PYZus{}data}\PY{o}{.}\PY{n}{shape}\PY{p}{)}
\end{Verbatim}


    \begin{Verbatim}[commandchars=\\\{\}]
('the Shape of The Dataset', (1000000, 8))

    \end{Verbatim}

    \begin{Verbatim}[commandchars=\\\{\}]
{\color{incolor}In [{\color{incolor}270}]:} \PY{c+c1}{\PYZsh{}chech how many  levels for each catagrical column}
          \PY{c+c1}{\PYZsh{}print(\PYZsq{}Columns:\PYZsq{},raw\PYZus{}data.columns.tolist())}
          \PY{k}{print}\PY{p}{(}\PY{l+s+s1}{\PYZsq{}}\PY{l+s+s1}{Number of unique IDs:}\PY{l+s+s1}{\PYZsq{}}\PY{p}{,} \PY{n+nb}{len}\PY{p}{(}\PY{n}{raw\PYZus{}data}\PY{o}{.}\PY{n}{ID}\PY{o}{.}\PY{n}{unique}\PY{p}{(}\PY{p}{)}\PY{p}{)}\PY{p}{)}
          \PY{k}{print}\PY{p}{(}\PY{l+s+s1}{\PYZsq{}}\PY{l+s+s1}{Number of unique Actions:}\PY{l+s+s1}{\PYZsq{}}\PY{p}{,} \PY{n+nb}{len}\PY{p}{(}\PY{n}{raw\PYZus{}data}\PY{o}{.}\PY{n}{Action}\PY{o}{.}\PY{n}{unique}\PY{p}{(}\PY{p}{)}\PY{p}{)}\PY{p}{)}
          \PY{k}{print}\PY{p}{(}\PY{l+s+s1}{\PYZsq{}}\PY{l+s+s1}{Number of unique Websites:}\PY{l+s+s1}{\PYZsq{}}\PY{p}{,} \PY{n+nb}{len}\PY{p}{(}\PY{n}{raw\PYZus{}data}\PY{o}{.}\PY{n}{Website}\PY{o}{.}\PY{n}{unique}\PY{p}{(}\PY{p}{)}\PY{p}{)}\PY{p}{)}
          \PY{k}{print}\PY{p}{(}\PY{l+s+s1}{\PYZsq{}}\PY{l+s+s1}{Number of unique BannerSizes:}\PY{l+s+s1}{\PYZsq{}}\PY{p}{,} \PY{n+nb}{len}\PY{p}{(}\PY{n}{raw\PYZus{}data}\PY{o}{.}\PY{n}{BannerSize}\PY{o}{.}\PY{n}{unique}\PY{p}{(}\PY{p}{)}\PY{p}{)}\PY{p}{)}
          \PY{k}{print}\PY{p}{(}\PY{l+s+s1}{\PYZsq{}}\PY{l+s+s1}{Number of unique Brands:}\PY{l+s+s1}{\PYZsq{}}\PY{p}{,} \PY{n+nb}{len}\PY{p}{(}\PY{n}{raw\PYZus{}data}\PY{o}{.}\PY{n}{Brand}\PY{o}{.}\PY{n}{unique}\PY{p}{(}\PY{p}{)}\PY{p}{)}\PY{p}{)}
          \PY{k}{print}\PY{p}{(}\PY{l+s+s1}{\PYZsq{}}\PY{l+s+s1}{Number of unique colours:}\PY{l+s+s1}{\PYZsq{}}\PY{p}{,} \PY{n+nb}{len}\PY{p}{(}\PY{n}{raw\PYZus{}data}\PY{o}{.}\PY{n}{colour}\PY{o}{.}\PY{n}{unique}\PY{p}{(}\PY{p}{)}\PY{p}{)}\PY{p}{)}
\end{Verbatim}


    \begin{Verbatim}[commandchars=\\\{\}]
('Number of unique IDs:', 218686)
('Number of unique Actions:', 2)
('Number of unique Websites:', 16)
('Number of unique BannerSizes:', 15)
('Number of unique Brands:', 10)
('Number of unique colours:', 7)

    \end{Verbatim}

    \subsection{The following plots show histograms of catagrical
features:}\label{the-following-plots-show-histograms-of-catagrical-features}

    \begin{Verbatim}[commandchars=\\\{\}]
{\color{incolor}In [{\color{incolor}271}]:} \PY{c+c1}{\PYZsh{} As \PYZsq{}Action\PYZsq{} attribute here is the label of our dataset, it is necessary to check if this dataset is balanced?}
          
          \PY{n}{sns}\PY{o}{.}\PY{n}{countplot}\PY{p}{(}\PY{n}{raw\PYZus{}data}\PY{o}{.}\PY{n}{Action}\PY{p}{)}
          \PY{c+c1}{\PYZsh{} One interesting thing is to figure out how many of records has Action attribute equal \PYZsq{}Click\PYZsq{}}
          \PY{n}{raw\PYZus{}data}\PY{o}{.}\PY{n}{groupby}\PY{p}{(}\PY{l+s+s1}{\PYZsq{}}\PY{l+s+s1}{Action}\PY{l+s+s1}{\PYZsq{}}\PY{p}{)}\PY{o}{.}\PY{n}{size}\PY{p}{(}\PY{p}{)}
\end{Verbatim}


\begin{Verbatim}[commandchars=\\\{\}]
{\color{outcolor}Out[{\color{outcolor}271}]:} Action
          Click       244
          View     999756
          dtype: int64
\end{Verbatim}
            
    \begin{center}
    \adjustimage{max size={0.9\linewidth}{0.9\paperheight}}{output_11_1.png}
    \end{center}
    { \hspace*{\fill} \\}
    
    \paragraph{This figure shows that that the dataset we have is highly
imbalanced}\label{this-figure-shows-that-that-the-dataset-we-have-is-highly-imbalanced}

The most possible action users taken is View, Just a few of them
actually click on ads. Therefore, we might need to look into that when
sampling training set and testing set. We need to keep this in mind when
performing modelling and evaluation.

    \begin{Verbatim}[commandchars=\\\{\}]
{\color{incolor}In [{\color{incolor}272}]:} \PY{n}{fig}\PY{o}{=}\PY{n}{plt}\PY{o}{.}\PY{n}{figure}\PY{p}{(}\PY{n}{figsize}\PY{o}{=}\PY{p}{(}\PY{l+m+mi}{15}\PY{p}{,}\PY{l+m+mi}{5}\PY{p}{)}\PY{p}{)}
          \PY{n}{sns}\PY{o}{.}\PY{n}{countplot}\PY{p}{(}\PY{n}{raw\PYZus{}data}\PY{o}{.}\PY{n}{Brand}\PY{p}{)}
\end{Verbatim}


\begin{Verbatim}[commandchars=\\\{\}]
{\color{outcolor}Out[{\color{outcolor}272}]:} <matplotlib.axes.\_subplots.AxesSubplot at 0x7f4c93154c10>
\end{Verbatim}
            
    \begin{center}
    \adjustimage{max size={0.9\linewidth}{0.9\paperheight}}{output_13_1.png}
    \end{center}
    { \hspace*{\fill} \\}
    
    \subsubsection{As for the 'brand' feature, we cannot find somthing that
we need topay attention
to}\label{as-for-the-brand-feature-we-cannot-find-somthing-that-we-need-topay-attention-to}

    \begin{Verbatim}[commandchars=\\\{\}]
{\color{incolor}In [{\color{incolor}273}]:} \PY{n}{fig}\PY{o}{=}\PY{n}{plt}\PY{o}{.}\PY{n}{figure}\PY{p}{(}\PY{n}{figsize}\PY{o}{=}\PY{p}{(}\PY{l+m+mi}{15}\PY{p}{,}\PY{l+m+mi}{5}\PY{p}{)}\PY{p}{)}
          \PY{n}{sns}\PY{o}{.}\PY{n}{countplot}\PY{p}{(}\PY{n}{raw\PYZus{}data}\PY{o}{.}\PY{n}{Website}\PY{p}{)}
\end{Verbatim}


\begin{Verbatim}[commandchars=\\\{\}]
{\color{outcolor}Out[{\color{outcolor}273}]:} <matplotlib.axes.\_subplots.AxesSubplot at 0x7f4c1a9c0ad0>
\end{Verbatim}
            
    \begin{center}
    \adjustimage{max size={0.9\linewidth}{0.9\paperheight}}{output_15_1.png}
    \end{center}
    { \hspace*{\fill} \\}
    
    \subsubsection{We can find that the majority of the ads were exposed on
Facebook according to our
dataset}\label{we-can-find-that-the-majority-of-the-ads-were-exposed-on-facebook-according-to-our-dataset}

    \begin{Verbatim}[commandchars=\\\{\}]
{\color{incolor}In [{\color{incolor}274}]:} \PY{n}{fig}\PY{o}{=}\PY{n}{plt}\PY{o}{.}\PY{n}{figure}\PY{p}{(}\PY{n}{figsize}\PY{o}{=}\PY{p}{(}\PY{l+m+mi}{15}\PY{p}{,}\PY{l+m+mi}{5}\PY{p}{)}\PY{p}{)}
          \PY{n}{sns}\PY{o}{.}\PY{n}{countplot}\PY{p}{(}\PY{n}{raw\PYZus{}data}\PY{o}{.}\PY{n}{BannerSize}\PY{p}{)}
\end{Verbatim}


\begin{Verbatim}[commandchars=\\\{\}]
{\color{outcolor}Out[{\color{outcolor}274}]:} <matplotlib.axes.\_subplots.AxesSubplot at 0x7f4c1aa1f3d0>
\end{Verbatim}
            
    \begin{center}
    \adjustimage{max size={0.9\linewidth}{0.9\paperheight}}{output_17_1.png}
    \end{center}
    { \hspace*{\fill} \\}
    
    \paragraph{Note here, there are two factors{[}other, Other{]}, they
should represents the same
meaning}\label{note-here-there-are-two-factorsother-other-they-should-represents-the-same-meaning}

This will be solved later

    \begin{Verbatim}[commandchars=\\\{\}]
{\color{incolor}In [{\color{incolor}275}]:} \PY{c+c1}{\PYZsh{}fig=plt.figure(figsize=(20,10))}
          \PY{n}{sns}\PY{o}{.}\PY{n}{countplot}\PY{p}{(}\PY{n}{raw\PYZus{}data}\PY{o}{.}\PY{n}{colour}\PY{p}{)}
\end{Verbatim}


\begin{Verbatim}[commandchars=\\\{\}]
{\color{outcolor}Out[{\color{outcolor}275}]:} <matplotlib.axes.\_subplots.AxesSubplot at 0x7f4c19324910>
\end{Verbatim}
            
    \begin{center}
    \adjustimage{max size={0.9\linewidth}{0.9\paperheight}}{output_19_1.png}
    \end{center}
    { \hspace*{\fill} \\}
    
    \subsubsection{As for the 'Colour' feature, we cannot find somthing that
we need topay attention
to.}\label{as-for-the-colour-feature-we-cannot-find-somthing-that-we-need-topay-attention-to.}

    \subsection{Several Observations}\label{several-observations}

From above results, few things can be found:

\begin{enumerate}
\def\labelenumi{\arabic{enumi}.}
\item
  The Number of unique IDs is greater than the number of our records,
  which means one user might view ads more than once. The dataset
  contains more than two records for a unique user.
\item
  At this stage, I discovered a problem, which is the BannerSizes
  feature, the intuition is that the bigger the banner size, there are
  higer probabilities that user will click on ads. Intrincally, it
  should not be catagrical feature, instead, it might be better to
  convert it to numberical feature, This is where I can do Feature
  Engineering later on.
\item
  The two features, colour and BannerSize contain 'other' catagory, this
  is not a big problem, Regaring to colour feature, we can regard the
  ''other catagory as a new catagory. Regaring to BannerSize feature, we
  will solve this when we perform feature engineering.
\item
  After looking at the distribution plots of catagorical features, we
  can conclude that there are no missing values in catagorical features
\end{enumerate}

    \section{Feature Engineering}\label{feature-engineering}

As mentioned before, BannerSize feature is a feature we might need to
introduce a feature-\/-'BannerArea'

    \begin{Verbatim}[commandchars=\\\{\}]
{\color{incolor}In [{\color{incolor}276}]:} \PY{c+c1}{\PYZsh{} fist we need to deal calculate BannerArea, we will regard \PYZsq{}other\PYZsq{} or \PYZsq{}Other\PYZsq{} factors as missing values in this column}
          \PY{n}{sizes}\PY{o}{=} \PY{n}{raw\PYZus{}data}\PY{o}{.}\PY{n}{BannerSize}\PY{o}{.}\PY{n}{unique}\PY{p}{(}\PY{p}{)}
          \PY{c+c1}{\PYZsh{} As there are just several possible banner sizes, we can calculate the possible area and then infer these values }
          \PY{c+c1}{\PYZsh{}to new BannerArea feature }
          \PY{c+c1}{\PYZsh{}define a method to calculate area of banner }
          \PY{n}{sizes} \PY{o}{=} \PY{n}{sizes}\PY{o}{.}\PY{n}{tolist}\PY{p}{(}\PY{p}{)}
          \PY{n}{new\PYZus{}dir} \PY{o}{=} \PY{p}{\PYZob{}}\PY{p}{\PYZcb{}}
          \PY{k}{for} \PY{n}{i}\PY{p}{,} \PY{n}{size} \PY{o+ow}{in} \PY{n+nb}{enumerate}\PY{p}{(}\PY{n}{sizes}\PY{p}{)}\PY{p}{:}
              \PY{k}{if} \PY{n}{size} \PY{o}{==}\PY{l+s+s1}{\PYZsq{}}\PY{l+s+s1}{Other}\PY{l+s+s1}{\PYZsq{}} \PY{o+ow}{or} \PY{n}{size} \PY{o}{==} \PY{l+s+s1}{\PYZsq{}}\PY{l+s+s1}{other}\PY{l+s+s1}{\PYZsq{}}\PY{p}{:}
                  \PY{n}{new\PYZus{}dir}\PY{p}{[}\PY{n}{i}\PY{p}{]}\PY{o}{=} \PY{l+m+mi}{0}  
              \PY{k}{else}\PY{p}{:}
                  \PY{n}{a}\PY{o}{=} \PY{n}{size}\PY{o}{.}\PY{n}{split}\PY{p}{(}\PY{l+s+s1}{\PYZsq{}}\PY{l+s+s1}{x}\PY{l+s+s1}{\PYZsq{}}\PY{p}{)}
                  \PY{n}{area} \PY{o}{=} \PY{n+nb}{int}\PY{p}{(}\PY{n}{a}\PY{p}{[}\PY{l+m+mi}{0}\PY{p}{]}\PY{p}{)}\PY{o}{*}\PY{n+nb}{int}\PY{p}{(}\PY{n}{a}\PY{p}{[}\PY{l+m+mi}{1}\PY{p}{]}\PY{p}{)}
                  \PY{n}{new\PYZus{}dir}\PY{p}{[}\PY{n}{i}\PY{p}{]} \PY{o}{=} \PY{n}{area}        
          \PY{k}{for} \PY{n}{i}\PY{p}{,} \PY{n}{size} \PY{o+ow}{in} \PY{n+nb}{enumerate}\PY{p}{(}\PY{n}{sizes}\PY{p}{)}\PY{p}{:}
              \PY{n}{mask} \PY{o}{=} \PY{n}{raw\PYZus{}data}\PY{o}{.}\PY{n}{BannerSize} \PY{o}{==} \PY{n}{size}
              \PY{n}{column\PYZus{}name} \PY{o}{=} \PY{l+s+s1}{\PYZsq{}}\PY{l+s+s1}{BannerSize}\PY{l+s+s1}{\PYZsq{}}
              \PY{n}{raw\PYZus{}data}\PY{o}{.}\PY{n}{loc}\PY{p}{[}\PY{n}{mask}\PY{p}{,} \PY{n}{column\PYZus{}name}\PY{p}{]} \PY{o}{=} \PY{n}{new\PYZus{}dir}\PY{p}{[}\PY{n}{i}\PY{p}{]}
              
          \PY{c+c1}{\PYZsh{} change the feayture name}
          \PY{n}{raw\PYZus{}data}\PY{o}{.}\PY{n}{rename}\PY{p}{(}\PY{n}{columns} \PY{o}{=} \PY{p}{\PYZob{}}\PY{l+s+s1}{\PYZsq{}}\PY{l+s+s1}{BannerSize}\PY{l+s+s1}{\PYZsq{}}\PY{p}{:} \PY{l+s+s1}{\PYZsq{}}\PY{l+s+s1}{BannerArea}\PY{l+s+s1}{\PYZsq{}}\PY{p}{\PYZcb{}}\PY{p}{,}\PY{n}{inplace} \PY{o}{=}\PY{n+nb+bp}{True}\PY{p}{)}   
          
          \PY{n}{raw\PYZus{}data}\PY{o}{.}\PY{n}{head}\PY{p}{(}\PY{p}{)}
\end{Verbatim}


\begin{Verbatim}[commandchars=\\\{\}]
{\color{outcolor}Out[{\color{outcolor}276}]:}                              ID        ActionTime Action        Website  \textbackslash{}
          0  AMsySZbNg8SBwPtj7IGHnip\_8aY0  1508761859495365   View       Facebook   
          1  AMsySZZcUFNGOJs-lfC9j-ZDlV7Z  1502847469736117   View       Facebook   
          2  AMsySZYoK8\_bD0hGv4zu0iRn1TFo  1514169431528120   View       Facebook   
          3  AMsySZahxpwLOWnMJj6RGATgwEOc  1515527952591896   View       Facebook   
          4  AMsySZbqgfMBDtAqZz1jVRaOmX00  1507514809374045   View  instagram.com   
          
            BannerArea     Brand  colour  InteractionTime  
          0     180000  Carter's    Pink         0.171601  
          1      75000  Burberry     Red         4.303945  
          2      65520    Disney  Yellow         0.868106  
          3      16000  Carter's  Yellow         0.802647  
          4      75000       Gap    Gold         0.627148  
\end{Verbatim}
            
    \begin{Verbatim}[commandchars=\\\{\}]
{\color{incolor}In [{\color{incolor}277}]:} \PY{c+c1}{\PYZsh{}df[[\PYZsq{}two\PYZsq{}, \PYZsq{}three\PYZsq{}]] = df[[\PYZsq{}two\PYZsq{}, \PYZsq{}three\PYZsq{}]].astype(float)}
          \PY{n}{raw\PYZus{}data}\PY{p}{[}\PY{p}{[}\PY{l+s+s1}{\PYZsq{}}\PY{l+s+s1}{BannerArea}\PY{l+s+s1}{\PYZsq{}}\PY{p}{]}\PY{p}{]} \PY{o}{=} \PY{n}{raw\PYZus{}data}\PY{p}{[}\PY{p}{[}\PY{l+s+s1}{\PYZsq{}}\PY{l+s+s1}{BannerArea}\PY{l+s+s1}{\PYZsq{}}\PY{p}{]}\PY{p}{]}\PY{o}{.}\PY{n}{astype}\PY{p}{(}\PY{n+nb}{float}\PY{p}{)}
          
          \PY{n}{raw\PYZus{}data}\PY{o}{.}\PY{n}{loc}\PY{p}{[}\PY{n}{raw\PYZus{}data}\PY{p}{[}\PY{l+s+s1}{\PYZsq{}}\PY{l+s+s1}{BannerArea}\PY{l+s+s1}{\PYZsq{}}\PY{p}{]} \PY{o}{==} \PY{l+m+mi}{0}\PY{p}{,}\PY{l+s+s1}{\PYZsq{}}\PY{l+s+s1}{BannerArea}\PY{l+s+s1}{\PYZsq{}}\PY{p}{]} \PY{o}{=} \PY{n}{np}\PY{o}{.}\PY{n}{nan}
\end{Verbatim}


    \begin{Verbatim}[commandchars=\\\{\}]
{\color{incolor}In [{\color{incolor}278}]:} \PY{n}{raw\PYZus{}data}\PY{o}{.}\PY{n}{BannerArea}\PY{o}{.}\PY{n}{isnull}\PY{p}{(}\PY{p}{)}\PY{o}{.}\PY{n}{unique}\PY{p}{(}\PY{p}{)}
\end{Verbatim}


\begin{Verbatim}[commandchars=\\\{\}]
{\color{outcolor}Out[{\color{outcolor}278}]:} array([False,  True], dtype=bool)
\end{Verbatim}
            
    \section{Step 2: Missing value checking and dealing with missing
values}\label{step-2-missing-value-checking-and-dealing-with-missing-values}

Defined a finction to check of there are missing values in a cenrtain
feature

    \begin{Verbatim}[commandchars=\\\{\}]
{\color{incolor}In [{\color{incolor}279}]:} \PY{k}{def} \PY{n+nf}{CheckMissingValues}\PY{p}{(}\PY{n}{feature}\PY{p}{)}\PY{p}{:}
              \PY{n}{cc} \PY{o}{=}\PY{n}{raw\PYZus{}data}\PY{p}{[}\PY{n}{feature}\PY{p}{]}\PY{o}{.}\PY{n}{isnull}\PY{p}{(}\PY{p}{)}\PY{o}{.}\PY{n}{unique}\PY{p}{(}\PY{p}{)}
              \PY{k}{if} \PY{n+nb}{len}\PY{p}{(}\PY{n}{cc}\PY{p}{)} \PY{o}{==} \PY{l+m+mi}{1}\PY{p}{:} \PY{c+c1}{\PYZsh{}if length=1 means return only false,so there are no misssing values}
                  \PY{k}{print} \PY{l+s+s1}{\PYZsq{}}\PY{l+s+s1}{There are no missing values in feature: }\PY{l+s+si}{\PYZpc{}s}\PY{l+s+s1}{\PYZsq{}}\PY{o}{\PYZpc{}}\PY{k}{feature}
              \PY{k}{else}\PY{p}{:}
                  \PY{k}{print} \PY{l+s+s1}{\PYZsq{}}\PY{l+s+s1}{Alert! There are missing values in feature:  }\PY{l+s+si}{\PYZpc{}s}\PY{l+s+s1}{\PYZsq{}}\PY{o}{\PYZpc{}}\PY{k}{feature}
          
          \PY{n}{column\PYZus{}list} \PY{o}{=} \PY{n}{raw\PYZus{}data}\PY{o}{.}\PY{n}{columns}\PY{o}{.}\PY{n}{tolist}\PY{p}{(}\PY{p}{)}
          \PY{c+c1}{\PYZsh{} check if feature has missing values}
          \PY{k}{for} \PY{n}{i} \PY{o+ow}{in} \PY{n}{column\PYZus{}list}\PY{p}{:}
              \PY{n}{CheckMissingValues}\PY{p}{(}\PY{n}{i}\PY{p}{)}
\end{Verbatim}


    \begin{Verbatim}[commandchars=\\\{\}]
There are no missing values in feature: ID
There are no missing values in feature: ActionTime
There are no missing values in feature: Action
There are no missing values in feature: Website
Alert! There are missing values in feature:  BannerArea
There are no missing values in feature: Brand
There are no missing values in feature: colour
Alert! There are missing values in feature:  InteractionTime

    \end{Verbatim}

    \paragraph{Addressing problems with NaN in the
data}\label{addressing-problems-with-nan-in-the-data}

As we saw from our EDA there were NaN in feature: InteractionTime. Our
model won't know what to do with these so we need to replace them with
something sensible.

There are quite a few options we can use - the mean, median,
most\_frequent, or a numeric value like 0. Playing with these will give
different results, for now I have it set to use the mean.

This uses the mean of the column in which the missing value is located.

    \begin{Verbatim}[commandchars=\\\{\}]
{\color{incolor}In [{\color{incolor}280}]:} \PY{c+c1}{\PYZsh{} Create a list of columns that have missing values and an index (True / False)}
          \PY{n}{df\PYZus{}missing} \PY{o}{=} \PY{n}{raw\PYZus{}data}\PY{o}{.}\PY{n}{isnull}\PY{p}{(}\PY{p}{)}\PY{o}{.}\PY{n}{sum}\PY{p}{(}\PY{n}{axis} \PY{o}{=} \PY{l+m+mi}{0}\PY{p}{)}\PY{o}{.}\PY{n}{reset\PYZus{}index}\PY{p}{(}\PY{p}{)}
          \PY{n}{df\PYZus{}missing}\PY{o}{.}\PY{n}{columns} \PY{o}{=} \PY{p}{[}\PY{l+s+s1}{\PYZsq{}}\PY{l+s+s1}{column\PYZus{}name}\PY{l+s+s1}{\PYZsq{}}\PY{p}{,} \PY{l+s+s1}{\PYZsq{}}\PY{l+s+s1}{missing\PYZus{}count}\PY{l+s+s1}{\PYZsq{}}\PY{p}{]}
          \PY{n}{idx\PYZus{}} \PY{o}{=} \PY{n}{df\PYZus{}missing}\PY{p}{[}\PY{l+s+s1}{\PYZsq{}}\PY{l+s+s1}{missing\PYZus{}count}\PY{l+s+s1}{\PYZsq{}}\PY{p}{]} \PY{o}{\PYZgt{}} \PY{l+m+mi}{0}
          \PY{n}{df\PYZus{}missing} \PY{o}{=} \PY{n}{df\PYZus{}missing}\PY{o}{.}\PY{n}{loc}\PY{p}{[}\PY{n}{idx\PYZus{}}\PY{p}{]}
          
          \PY{n}{cols\PYZus{}missing} \PY{o}{=} \PY{n}{df\PYZus{}missing}\PY{o}{.}\PY{n}{column\PYZus{}name}\PY{o}{.}\PY{n}{values}
          \PY{n}{idx\PYZus{}cols\PYZus{}missing} \PY{o}{=} \PY{n}{raw\PYZus{}data}\PY{o}{.}\PY{n}{columns}\PY{o}{.}\PY{n}{isin}\PY{p}{(}\PY{n}{cols\PYZus{}missing}\PY{p}{)}
          \PY{k}{print} \PY{n}{df\PYZus{}missing} \PY{c+c1}{\PYZsh{} this will count the number of missing values in our dataset}
\end{Verbatim}


    \begin{Verbatim}[commandchars=\\\{\}]
       column\_name  missing\_count
4       BannerArea         162246
7  InteractionTime              8

    \end{Verbatim}

    As there are only 8 missing values in InteractionTime, Therefore, the
way of dealing with missing values won't affect prediction results too
much

    \begin{Verbatim}[commandchars=\\\{\}]
{\color{incolor}In [{\color{incolor}281}]:} \PY{c+c1}{\PYZsh{} Instantiate an imputer}
          \PY{n}{imputer} \PY{o}{=} \PY{n}{preprocessing}\PY{o}{.}\PY{n}{Imputer}\PY{p}{(}\PY{n}{missing\PYZus{}values}\PY{o}{=}\PY{l+s+s1}{\PYZsq{}}\PY{l+s+s1}{NaN}\PY{l+s+s1}{\PYZsq{}}\PY{p}{,} \PY{n}{strategy} \PY{o}{=} \PY{l+s+s1}{\PYZsq{}}\PY{l+s+s1}{mean}\PY{l+s+s1}{\PYZsq{}}\PY{p}{,} \PY{n}{axis} \PY{o}{=} \PY{l+m+mi}{0}\PY{p}{)}
          
          \PY{c+c1}{\PYZsh{} Fit the imputer using all of our data (but not any dates)}
          \PY{n}{imputer}\PY{o}{.}\PY{n}{fit}\PY{p}{(}\PY{n}{raw\PYZus{}data}\PY{o}{.}\PY{n}{loc}\PY{p}{[}\PY{p}{:}\PY{p}{,} \PY{n}{idx\PYZus{}cols\PYZus{}missing}\PY{p}{]}\PY{p}{)}
          
          \PY{c+c1}{\PYZsh{} Apply the imputer}
          \PY{n}{raw\PYZus{}data}\PY{o}{.}\PY{n}{loc}\PY{p}{[}\PY{p}{:}\PY{p}{,} \PY{n}{idx\PYZus{}cols\PYZus{}missing}\PY{p}{]} \PY{o}{=} \PY{n}{imputer}\PY{o}{.}\PY{n}{transform}\PY{p}{(}\PY{n}{raw\PYZus{}data}\PY{o}{.}\PY{n}{loc}\PY{p}{[}\PY{p}{:}\PY{p}{,} \PY{n}{idx\PYZus{}cols\PYZus{}missing}\PY{p}{]}\PY{p}{)}
\end{Verbatim}


    \begin{Verbatim}[commandchars=\\\{\}]
{\color{incolor}In [{\color{incolor}282}]:} \PY{c+c1}{\PYZsh{}check again if missing values are filled already?}
          \PY{k}{print} \PY{l+s+s1}{\PYZsq{}}\PY{l+s+s1}{After dealing with missing values}\PY{l+s+s1}{\PYZsq{}}
          \PY{n}{CheckMissingValues}\PY{p}{(}\PY{l+s+s1}{\PYZsq{}}\PY{l+s+s1}{InteractionTime}\PY{l+s+s1}{\PYZsq{}}\PY{p}{)}
          \PY{n}{CheckMissingValues}\PY{p}{(}\PY{l+s+s1}{\PYZsq{}}\PY{l+s+s1}{BannerArea}\PY{l+s+s1}{\PYZsq{}}\PY{p}{)}
          \PY{c+c1}{\PYZsh{}let\PYZsq{}s take a look at the histogram of the two numerical data }
          \PY{c+c1}{\PYZsh{} let\PYZsq{}s take a look at the distrinution of data}
\end{Verbatim}


    \begin{Verbatim}[commandchars=\\\{\}]
After dealing with missing values
There are no missing values in feature: InteractionTime
There are no missing values in feature: BannerArea

    \end{Verbatim}

    \begin{Verbatim}[commandchars=\\\{\}]
{\color{incolor}In [{\color{incolor}283}]:} \PY{k+kn}{import} \PY{n+nn}{matplotlib.pyplot} \PY{k+kn}{as} \PY{n+nn}{plt}
          \PY{c+c1}{\PYZsh{} plot a line, implicitly creating a subplot(111)}
          
          
          \PY{c+c1}{\PYZsh{} now create a subplot which represents the top plot of a grid}
          \PY{c+c1}{\PYZsh{} with 2 rows and 1 column. Since this subplot will overlap the}
          \PY{c+c1}{\PYZsh{} first, the plot (and its axes) previously created, will be removed}
          \PY{n}{plt}\PY{o}{.}\PY{n}{subplot}\PY{p}{(}\PY{l+m+mi}{212}\PY{p}{)}
          \PY{n}{plt}\PY{o}{.}\PY{n}{hist}\PY{p}{(}\PY{n}{raw\PYZus{}data}\PY{p}{[}\PY{l+s+s1}{\PYZsq{}}\PY{l+s+s1}{InteractionTime}\PY{l+s+s1}{\PYZsq{}}\PY{p}{]}\PY{p}{)}\PY{p}{;}
          \PY{n}{plt}\PY{o}{.}\PY{n}{subplot}\PY{p}{(}\PY{l+m+mi}{211}\PY{p}{)} \PY{c+c1}{\PYZsh{} creates 2nd subplot with yellow background}
          \PY{n}{plt}\PY{o}{.}\PY{n}{hist}\PY{p}{(}\PY{n}{raw\PYZus{}data}\PY{p}{[}\PY{l+s+s1}{\PYZsq{}}\PY{l+s+s1}{ActionTime}\PY{l+s+s1}{\PYZsq{}}\PY{p}{]}\PY{p}{)}\PY{p}{;}
\end{Verbatim}


    \begin{center}
    \adjustimage{max size={0.9\linewidth}{0.9\paperheight}}{output_33_0.png}
    \end{center}
    { \hspace*{\fill} \\}
    
    \paragraph{This figure shows the histograms of ActionTime and
Interaction
time}\label{this-figure-shows-the-histograms-of-actiontime-and-interaction-time}

    \subsection{Encoding categorical
features}\label{encoding-categorical-features}

We will take a naive approach and assign a numeric value to each
categorical feature in our dataset. Sklearn's preprocessing unit has a
tool called LabelEncoder() which can do just that for us.

    \begin{Verbatim}[commandchars=\\\{\}]
{\color{incolor}In [{\color{incolor}284}]:} \PY{c+c1}{\PYZsh{} extract the label column:}
          \PY{n}{labels} \PY{o}{=} \PY{n}{raw\PYZus{}data}\PY{p}{[}\PY{p}{[}\PY{l+s+s1}{\PYZsq{}}\PY{l+s+s1}{Action}\PY{l+s+s1}{\PYZsq{}}\PY{p}{]}\PY{p}{]}
          
          \PY{n}{df\PYZus{}fea} \PY{o}{=} \PY{n}{raw\PYZus{}data}\PY{o}{.}\PY{n}{drop}\PY{p}{(}\PY{p}{[}\PY{l+s+s2}{\PYZdq{}}\PY{l+s+s2}{Action}\PY{l+s+s2}{\PYZdq{}}\PY{p}{]}\PY{p}{,} \PY{n}{axis}\PY{o}{=}\PY{l+m+mi}{1}\PY{p}{)}
          \PY{c+c1}{\PYZsh{} encoding categorical features}
          \PY{k}{for} \PY{n}{c} \PY{o+ow}{in} \PY{n}{df\PYZus{}fea}\PY{o}{.}\PY{n}{columns}\PY{p}{:}
              \PY{k}{if} \PY{n}{df\PYZus{}fea}\PY{p}{[}\PY{n}{c}\PY{p}{]}\PY{o}{.}\PY{n}{dtype} \PY{o}{==} \PY{l+s+s1}{\PYZsq{}}\PY{l+s+s1}{object}\PY{l+s+s1}{\PYZsq{}}\PY{p}{:}
                  \PY{n}{lbl} \PY{o}{=} \PY{n}{preprocessing}\PY{o}{.}\PY{n}{LabelEncoder}\PY{p}{(}\PY{p}{)}
                  \PY{n}{lbl}\PY{o}{.}\PY{n}{fit}\PY{p}{(}\PY{n+nb}{list}\PY{p}{(}\PY{n}{df\PYZus{}fea}\PY{p}{[}\PY{n}{c}\PY{p}{]}\PY{o}{.}\PY{n}{values}\PY{p}{)}\PY{p}{)} 
                  \PY{n}{df\PYZus{}fea}\PY{p}{[}\PY{n}{c}\PY{p}{]} \PY{o}{=} \PY{n}{lbl}\PY{o}{.}\PY{n}{transform}\PY{p}{(}\PY{n+nb}{list}\PY{p}{(}\PY{n}{df\PYZus{}fea}\PY{p}{[}\PY{n}{c}\PY{p}{]}\PY{o}{.}\PY{n}{values}\PY{p}{)}\PY{p}{)}
          \PY{c+c1}{\PYZsh{}encoding label column: click=1,view =0}
          \PY{n}{labels} \PY{o}{=} \PY{n}{labels}\PY{o}{.}\PY{n}{Action}\PY{o}{.}\PY{n}{map}\PY{p}{(}\PY{n+nb}{dict}\PY{p}{(}\PY{n}{Click}\PY{o}{=}\PY{l+m+mi}{1}\PY{p}{,} \PY{n}{View}\PY{o}{=}\PY{l+m+mi}{0}\PY{p}{)}\PY{p}{)}\PY{o}{.}\PY{n}{to\PYZus{}frame}\PY{p}{(}\PY{p}{)}\PY{p}{;}
          
          \PY{k}{print} \PY{n+nb}{type}\PY{p}{(}\PY{n}{labels}\PY{p}{)}
          \PY{k}{print} \PY{n}{df\PYZus{}fea}\PY{o}{.}\PY{n}{shape}
          \PY{n}{df\PYZus{}fea}\PY{o}{.}\PY{n}{head}\PY{p}{(}\PY{p}{)}
          
          \PY{c+c1}{\PYZsh{}split the data into train set and test set}
\end{Verbatim}


    \begin{Verbatim}[commandchars=\\\{\}]
<class 'pandas.core.frame.DataFrame'>
(1000000, 7)

    \end{Verbatim}

\begin{Verbatim}[commandchars=\\\{\}]
{\color{outcolor}Out[{\color{outcolor}284}]:}        ID        ActionTime  Website  BannerArea  Brand  colour  \textbackslash{}
          0  185287  1508761859495365        5    180000.0      2       2   
          1   89056  1502847469736117        5     75000.0      1       3   
          2   44526  1514169431528120        5     65520.0      3       5   
          3  148601  1515527952591896        5     16000.0      2       5   
          4  210803  1507514809374045       15     75000.0      4       1   
          
             InteractionTime  
          0         0.171601  
          1         4.303945  
          2         0.868106  
          3         0.802647  
          4         0.627148  
\end{Verbatim}
            
    \subsection{Multivariate Analysis}\label{multivariate-analysis}

\paragraph{Use the correlation matrix to check if data has
multicolinearity}\label{use-the-correlation-matrix-to-check-if-data-has-multicolinearity}

    \begin{Verbatim}[commandchars=\\\{\}]
{\color{incolor}In [{\color{incolor}285}]:} \PY{c+c1}{\PYZsh{}Correlation Matrix}
          \PY{n}{corr} \PY{o}{=} \PY{n}{df\PYZus{}fea}\PY{o}{.}\PY{n}{corr}\PY{p}{(}\PY{p}{)}
          \PY{n}{corr} \PY{o}{=} \PY{p}{(}\PY{n}{corr}\PY{p}{)}
          \PY{n}{sns}\PY{o}{.}\PY{n}{heatmap}\PY{p}{(}\PY{n}{corr}\PY{p}{,} 
                      \PY{n}{xticklabels}\PY{o}{=}\PY{n}{corr}\PY{o}{.}\PY{n}{columns}\PY{o}{.}\PY{n}{values}\PY{p}{,}
                      \PY{n}{yticklabels}\PY{o}{=}\PY{n}{corr}\PY{o}{.}\PY{n}{columns}\PY{o}{.}\PY{n}{values}\PY{p}{)}
          \PY{n}{plt}\PY{o}{.}\PY{n}{title}\PY{p}{(}\PY{l+s+s1}{\PYZsq{}}\PY{l+s+s1}{Heatmap of Correlation Matrix}\PY{l+s+s1}{\PYZsq{}}\PY{p}{)}
          \PY{n}{corr}
\end{Verbatim}


\begin{Verbatim}[commandchars=\\\{\}]
{\color{outcolor}Out[{\color{outcolor}285}]:}                        ID  ActionTime   Website  BannerArea     Brand  \textbackslash{}
          ID               1.000000    0.005492 -0.014643   -0.005110  0.000837   
          ActionTime       0.005492    1.000000 -0.072466   -0.136278  0.000056   
          Website         -0.014643   -0.072466  1.000000    0.296087 -0.001241   
          BannerArea      -0.005110   -0.136278  0.296087    1.000000 -0.002021   
          Brand            0.000837    0.000056 -0.001241   -0.002021  1.000000   
          colour          -0.000189    0.000037  0.000620    0.000187  0.001227   
          InteractionTime  0.000294   -0.000396  0.001795   -0.000381  0.000871   
          
                             colour  InteractionTime  
          ID              -0.000189         0.000294  
          ActionTime       0.000037        -0.000396  
          Website          0.000620         0.001795  
          BannerArea       0.000187        -0.000381  
          Brand            0.001227         0.000871  
          colour           1.000000        -0.001400  
          InteractionTime -0.001400         1.000000  
\end{Verbatim}
            
    \begin{center}
    \adjustimage{max size={0.9\linewidth}{0.9\paperheight}}{output_38_1.png}
    \end{center}
    { \hspace*{\fill} \\}
    
    \paragraph{This correlation matrix shows there features are not
correlated to each other, which is good, we might not need to deal with
correlated
features}\label{this-correlation-matrix-shows-there-features-are-not-correlated-to-each-other-which-is-good-we-might-not-need-to-deal-with-correlated-features}

    \subsection{Split Dataset into train set and test set by using
stratified
sampling}\label{split-dataset-into-train-set-and-test-set-by-using-stratified-sampling}

As the dataset is imbalanced and we want to evaluate the model
performance on both 'Action' equally, Therefore, dividing the dataset
into train and test sets using stratified sampling is necessary

Here, 80\% of the data are used for training models and the rest of them
are used for testing

    \begin{Verbatim}[commandchars=\\\{\}]
{\color{incolor}In [{\color{incolor}286}]:} \PY{c+c1}{\PYZsh{} split data into train set and test set;}
          \PY{n}{X\PYZus{}train}\PY{p}{,} \PY{n}{X\PYZus{}test}\PY{p}{,} \PY{n}{y\PYZus{}train}\PY{p}{,} \PY{n}{y\PYZus{}test} \PY{o}{=} \PY{n}{train\PYZus{}test\PYZus{}split}\PY{p}{(}\PY{n}{df\PYZus{}fea}\PY{p}{,} \PY{n}{labels}\PY{p}{,} \PY{n}{test\PYZus{}size}\PY{o}{=}\PY{l+m+mf}{0.2}\PY{p}{,} \PY{n}{stratify}\PY{o}{=}\PY{n}{labels}\PY{p}{)}
\end{Verbatim}


    \begin{Verbatim}[commandchars=\\\{\}]
{\color{incolor}In [{\color{incolor}287}]:} \PY{n+nb}{type}\PY{p}{(}\PY{n}{X\PYZus{}train}\PY{p}{)}
\end{Verbatim}


\begin{Verbatim}[commandchars=\\\{\}]
{\color{outcolor}Out[{\color{outcolor}287}]:} pandas.core.frame.DataFrame
\end{Verbatim}
            
    \section{Feature Selection}\label{feature-selection}

\subsubsection{At the very begining, we can use a random forest
classifier to do feature
selection}\label{at-the-very-begining-we-can-use-a-random-forest-classifier-to-do-feature-selection}

    \begin{Verbatim}[commandchars=\\\{\}]
{\color{incolor}In [{\color{incolor}288}]:} \PY{c+c1}{\PYZsh{} Define a classifier}
          \PY{n}{rf\PYZus{}clf} \PY{o}{=} \PY{n}{RandomForestClassifier}\PY{p}{(}\PY{n}{max\PYZus{}depth}\PY{o}{=}\PY{l+m+mi}{15}\PY{p}{,}\PY{n}{n\PYZus{}estimators}\PY{o}{=}\PY{l+m+mi}{70}\PY{p}{,} \PY{n}{min\PYZus{}samples\PYZus{}leaf}\PY{o}{=}\PY{l+m+mi}{50}\PY{p}{,}
                                            \PY{n}{min\PYZus{}samples\PYZus{}split}\PY{o}{=}\PY{l+m+mi}{100}\PY{p}{,} \PY{n}{random\PYZus{}state}\PY{o}{=}\PY{l+m+mi}{10}\PY{p}{)}
          
          \PY{c+c1}{\PYZsh{} Train the model}
          \PY{n}{rf\PYZus{}clf}\PY{o}{.}\PY{n}{fit}\PY{p}{(}\PY{n}{X\PYZus{}train}\PY{p}{,}\PY{n}{y\PYZus{}train}\PY{p}{)}
          
          \PY{c+c1}{\PYZsh{} Plot the top 40 important features}
          \PY{n}{imp\PYZus{}feat\PYZus{}rf} \PY{o}{=} \PY{n}{pd}\PY{o}{.}\PY{n}{Series}\PY{p}{(}\PY{n}{rf\PYZus{}clf}\PY{o}{.}\PY{n}{feature\PYZus{}importances\PYZus{}}\PY{p}{,} \PY{n}{index}\PY{o}{=}\PY{n}{X\PYZus{}train}\PY{o}{.}\PY{n}{columns}\PY{p}{)}\PY{o}{.}\PY{n}{sort\PYZus{}values}\PY{p}{(}\PY{n}{ascending}\PY{o}{=}\PY{n+nb+bp}{False}\PY{p}{)}
          \PY{n}{imp\PYZus{}feat\PYZus{}rf}\PY{o}{.}\PY{n}{plot}\PY{p}{(}\PY{n}{kind}\PY{o}{=}\PY{l+s+s1}{\PYZsq{}}\PY{l+s+s1}{bar}\PY{l+s+s1}{\PYZsq{}}\PY{p}{,} \PY{n}{title}\PY{o}{=}\PY{l+s+s1}{\PYZsq{}}\PY{l+s+s1}{Feature Importance with Random Forest}\PY{l+s+s1}{\PYZsq{}}\PY{p}{,} \PY{n}{figsize}\PY{o}{=}\PY{p}{(}\PY{l+m+mi}{12}\PY{p}{,}\PY{l+m+mi}{8}\PY{p}{)}\PY{p}{)}
          \PY{n}{plt}\PY{o}{.}\PY{n}{ylabel}\PY{p}{(}\PY{l+s+s1}{\PYZsq{}}\PY{l+s+s1}{Feature Importance values}\PY{l+s+s1}{\PYZsq{}}\PY{p}{)}
          \PY{n}{plt}\PY{o}{.}\PY{n}{subplots\PYZus{}adjust}\PY{p}{(}\PY{n}{bottom}\PY{o}{=}\PY{l+m+mf}{0.25}\PY{p}{)}
          \PY{n}{plt}\PY{o}{.}\PY{n}{savefig}\PY{p}{(}\PY{l+s+s1}{\PYZsq{}}\PY{l+s+s1}{FeatImportance.png}\PY{l+s+s1}{\PYZsq{}}\PY{p}{)}
          \PY{n}{plt}\PY{o}{.}\PY{n}{show}\PY{p}{(}\PY{p}{)}
\end{Verbatim}


    \begin{Verbatim}[commandchars=\\\{\}]
/home/lein/miniconda2/lib/python2.7/site-packages/ipykernel\_launcher.py:6: DataConversionWarning: A column-vector y was passed when a 1d array was expected. Please change the shape of y to (n\_samples,), for example using ravel().
  

    \end{Verbatim}

    \begin{center}
    \adjustimage{max size={0.9\linewidth}{0.9\paperheight}}{output_44_1.png}
    \end{center}
    { \hspace*{\fill} \\}
    
    \subsubsection{Observesion of importance of
features.}\label{observesion-of-importance-of-features.}

\begin{enumerate}
\def\labelenumi{\arabic{enumi}.}
\tightlist
\item
  we can find that ActionTime, InteractionTime and user ID are the top 3
  important features among all the features.\\
\item
  One interesting finding is website is an insignificant feature
  accoring to this figure, However, our intuition is the
  click-through-rate might be related to which wensite the ads were
  exposed because we need to spend more money on popular webisite like
  Facebook in order to put on ads.
\end{enumerate}

At this stage, we cannot conclude that website is a unimportant feature,
In order to get more confidence, Let us try few more feature selection
methods

    \subsubsection{Therefore I use XGBoost regressor to generate the rank of
"feature
importance"}\label{therefore-i-use-xgboost-regressor-to-generate-the-rank-of-feature-importance}

    \begin{Verbatim}[commandchars=\\\{\}]
{\color{incolor}In [{\color{incolor}289}]:} \PY{n}{xgb} \PY{o}{=} \PY{n}{XGBRegressor}\PY{p}{(}\PY{p}{)}
          \PY{n}{xgb}\PY{o}{.}\PY{n}{fit}\PY{p}{(}\PY{n}{X\PYZus{}train}\PY{p}{,} \PY{n}{y\PYZus{}train}\PY{p}{)}
          \PY{n}{imp} \PY{o}{=} \PY{n}{pd}\PY{o}{.}\PY{n}{DataFrame}\PY{p}{(}\PY{n}{xgb}\PY{o}{.}\PY{n}{feature\PYZus{}importances\PYZus{}} \PY{p}{,}\PY{n}{columns} \PY{o}{=} \PY{p}{[}\PY{l+s+s1}{\PYZsq{}}\PY{l+s+s1}{Importance}\PY{l+s+s1}{\PYZsq{}}\PY{p}{]}\PY{p}{,}\PY{n}{index} \PY{o}{=} \PY{n}{X\PYZus{}train}\PY{o}{.}\PY{n}{columns}\PY{p}{)}
          \PY{n}{imp} \PY{o}{=} \PY{n}{imp}\PY{o}{.}\PY{n}{sort\PYZus{}values}\PY{p}{(}\PY{p}{[}\PY{l+s+s1}{\PYZsq{}}\PY{l+s+s1}{Importance}\PY{l+s+s1}{\PYZsq{}}\PY{p}{]}\PY{p}{,} \PY{n}{ascending} \PY{o}{=} \PY{n+nb+bp}{False}\PY{p}{)}
          
          \PY{n}{imp}\PY{o}{.}\PY{n}{plot}\PY{p}{(}\PY{n}{kind}\PY{o}{=}\PY{l+s+s1}{\PYZsq{}}\PY{l+s+s1}{bar}\PY{l+s+s1}{\PYZsq{}}\PY{p}{,} \PY{n}{title}\PY{o}{=}\PY{l+s+s1}{\PYZsq{}}\PY{l+s+s1}{Feature Importance with XGBoost}\PY{l+s+s1}{\PYZsq{}}\PY{p}{,} \PY{n}{figsize}\PY{o}{=}\PY{p}{(}\PY{l+m+mi}{12}\PY{p}{,}\PY{l+m+mi}{8}\PY{p}{)}\PY{p}{)}
          \PY{n}{plt}\PY{o}{.}\PY{n}{ylabel}\PY{p}{(}\PY{l+s+s1}{\PYZsq{}}\PY{l+s+s1}{Feature Importance values}\PY{l+s+s1}{\PYZsq{}}\PY{p}{)}
          \PY{n}{plt}\PY{o}{.}\PY{n}{subplots\PYZus{}adjust}\PY{p}{(}\PY{n}{bottom}\PY{o}{=}\PY{l+m+mf}{0.25}\PY{p}{)}
          \PY{n}{plt}\PY{o}{.}\PY{n}{show}\PY{p}{(}\PY{p}{)}
\end{Verbatim}


    \begin{center}
    \adjustimage{max size={0.9\linewidth}{0.9\paperheight}}{output_47_0.png}
    \end{center}
    { \hspace*{\fill} \\}
    
    \paragraph{The feature importance results from XGBoost is a little bit
different, but the top three most important features are the
same,}\label{the-feature-importance-results-from-xgboost-is-a-little-bit-different-but-the-top-three-most-important-features-are-the-same}

\subsection{Now we can use RFECV to eliminate the redundant
features(i.e. Select
features).}\label{now-we-can-use-rfecv-to-eliminate-the-redundant-featuresi.e.-select-features.}

The following are what I did originally to select features. I ran a loop
and use XGBoost regressor with cross validation to see the improvement
of Root-Mean-Squared-Error as a function of the number of features
included. This method is essentially the RFECV algorithm).

    \begin{Verbatim}[commandchars=\\\{\}]
{\color{incolor}In [{\color{incolor}290}]:} \PY{c+c1}{\PYZsh{} Define a function to calculate RMSE}
          \PY{k}{def} \PY{n+nf}{rmse}\PY{p}{(}\PY{n}{y\PYZus{}true}\PY{p}{,} \PY{n}{y\PYZus{}pred}\PY{p}{)}\PY{p}{:}
              \PY{k}{return} \PY{n}{np}\PY{o}{.}\PY{n}{sqrt}\PY{p}{(}\PY{n}{np}\PY{o}{.}\PY{n}{mean}\PY{p}{(}\PY{p}{(}\PY{n}{y\PYZus{}true}\PY{o}{\PYZhy{}}\PY{n}{y\PYZus{}pred}\PY{p}{)}\PY{o}{*}\PY{o}{*}\PY{l+m+mi}{2}\PY{p}{)}\PY{p}{)}
          
          \PY{c+c1}{\PYZsh{} Define a function to calculate negative RMSE (as a score)}
          \PY{k}{def} \PY{n+nf}{nrmse}\PY{p}{(}\PY{n}{y\PYZus{}true}\PY{p}{,} \PY{n}{y\PYZus{}pred}\PY{p}{)}\PY{p}{:}
              \PY{k}{return} \PY{o}{\PYZhy{}}\PY{l+m+mf}{1.0}\PY{o}{*}\PY{n}{rmse}\PY{p}{(}\PY{n}{y\PYZus{}true}\PY{p}{,} \PY{n}{y\PYZus{}pred}\PY{p}{)}
          
          \PY{n}{neg\PYZus{}rmse} \PY{o}{=} \PY{n}{make\PYZus{}scorer}\PY{p}{(}\PY{n}{nrmse}\PY{p}{)}
          
          \PY{n}{estimator} \PY{o}{=} \PY{n}{XGBRegressor}\PY{p}{(}\PY{p}{)}
          \PY{n}{selector} \PY{o}{=} \PY{n}{RFECV}\PY{p}{(}\PY{n}{estimator}\PY{p}{,} \PY{n}{cv} \PY{o}{=} \PY{l+m+mi}{3}\PY{p}{,} \PY{n}{n\PYZus{}jobs} \PY{o}{=} \PY{o}{\PYZhy{}}\PY{l+m+mi}{1}\PY{p}{,} \PY{n}{scoring} \PY{o}{=} \PY{n}{neg\PYZus{}rmse}\PY{p}{)}
          \PY{n}{selector} \PY{o}{=} \PY{n}{selector}\PY{o}{.}\PY{n}{fit}\PY{p}{(}\PY{n}{X\PYZus{}train}\PY{p}{,} \PY{n}{y\PYZus{}train}\PY{p}{)}
          
          \PY{k}{print}\PY{p}{(}\PY{l+s+s2}{\PYZdq{}}\PY{l+s+s2}{The number of selected features is: \PYZob{}\PYZcb{}}\PY{l+s+s2}{\PYZdq{}}\PY{o}{.}\PY{n}{format}\PY{p}{(}\PY{n}{selector}\PY{o}{.}\PY{n}{n\PYZus{}features\PYZus{}}\PY{p}{)}\PY{p}{)}
          
          \PY{n}{features\PYZus{}kept} \PY{o}{=} \PY{n}{X\PYZus{}train}\PY{o}{.}\PY{n}{columns}\PY{o}{.}\PY{n}{values}\PY{p}{[}\PY{n}{selector}\PY{o}{.}\PY{n}{support\PYZus{}}\PY{p}{]} 
          \PY{k}{print} \PY{l+s+s1}{\PYZsq{}}\PY{l+s+s1}{the selected features are the followings}\PY{l+s+s1}{\PYZsq{}}
          \PY{k}{print} \PY{n}{features\PYZus{}kept}
          
          \PY{n}{X\PYZus{}train} \PY{o}{=} \PY{n}{selector}\PY{o}{.}\PY{n}{transform}\PY{p}{(}\PY{n}{X\PYZus{}train}\PY{p}{)}  
          \PY{n}{X\PYZus{}test} \PY{o}{=} \PY{n}{selector}\PY{o}{.}\PY{n}{transform}\PY{p}{(}\PY{n}{X\PYZus{}test}\PY{p}{)}
          \PY{k}{print} \PY{n+nb}{type}\PY{p}{(}\PY{n}{X\PYZus{}train}\PY{p}{)}
\end{Verbatim}


    \begin{Verbatim}[commandchars=\\\{\}]
The number of selected features is: 7
the selected features are the followings
['ID' 'ActionTime' 'Website' 'BannerArea' 'Brand' 'colour'
 'InteractionTime']
<type 'numpy.ndarray'>

    \end{Verbatim}

    \subsubsection{At this stage, the feature selection part is
done}\label{at-this-stage-the-feature-selection-part-is-done}

\section{Model Selection and Model
Evaluation}\label{model-selection-and-model-evaluation}

\subsection{-\/-Cross-validation with stratification based on class
labels.}\label{cross-validation-with-stratification-based-on-class-labels.}

One thing we need to clarify is that our goal is to estimate the
probability of click on ads. This a regression problem instead of
classifiction problem: 1. As our dataset almost consists of catagrical
features, and we know that tree-based method perform well on categrical
data

\begin{enumerate}
\def\labelenumi{\arabic{enumi}.}
\setcounter{enumi}{1}
\item
  we need to adopt cross-validation with stratification becasue of our
  imbalanced dataset
\item
  Therefore, I decided to implement three-basned regression models
  first(we can choose more than three models if we want, but three
  models are enough for now to show an example of model selection using
  cross-validation. beside, becasue of my computer's memory restriction,
  I am unable to perform cross-validation for more models)
\item
  After evaluate their performance with the averaged L2 loss through
  cross-validation, we can pick the one has lowest loss as our final
  model.
\item
  After performing the above, we will do hyperparameter tuning to
  achieve a lower loss.(better model)
\end{enumerate}

Let us try a decision tree first.

    \begin{Verbatim}[commandchars=\\\{\}]
{\color{incolor}In [{\color{incolor}291}]:} \PY{c+c1}{\PYZsh{} before get started, we need to split data into subsets. }
          \PY{c+c1}{\PYZsh{}hus, let us combine X\PYZus{}train, X\PYZus{}test, y\PYZus{}train and y\PYZus{}test together }
          
          \PY{n}{X}\PY{o}{=}\PY{n}{np}\PY{o}{.}\PY{n}{vstack}\PY{p}{(}\PY{p}{(}\PY{n}{X\PYZus{}train}\PY{p}{,}\PY{n}{X\PYZus{}test}\PY{p}{)}\PY{p}{)}
          \PY{n}{y}\PY{o}{=}\PY{n}{np}\PY{o}{.}\PY{n}{vstack}\PY{p}{(}\PY{p}{(}\PY{n}{y\PYZus{}train}\PY{p}{,}\PY{n}{y\PYZus{}test}\PY{p}{)}\PY{p}{)}
\end{Verbatim}


    \begin{Verbatim}[commandchars=\\\{\}]
{\color{incolor}In [{\color{incolor}238}]:} \PY{k+kn}{from} \PY{n+nn}{sklearn.model\PYZus{}selection} \PY{k+kn}{import} \PY{n}{StratifiedKFold}
          \PY{k+kn}{from} \PY{n+nn}{sklearn.ensemble} \PY{k+kn}{import} \PY{n}{GradientBoostingRegressor}
          \PY{k+kn}{from} \PY{n+nn}{sklearn.metrics} \PY{k+kn}{import} \PY{n}{mean\PYZus{}squared\PYZus{}error}
          \PY{k+kn}{from} \PY{n+nn}{sklearn} \PY{k+kn}{import} \PY{n}{tree}
          
          \PY{c+c1}{\PYZsh{}\PYZsh{} Start with stratified 10 folde cross\PYZhy{}validation for }
          \PY{n}{skf} \PY{o}{=} \PY{n}{StratifiedKFold}\PY{p}{(}\PY{n}{n\PYZus{}splits}\PY{o}{=}\PY{l+m+mi}{10}\PY{p}{)}
          \PY{n}{tree\PYZus{}mse\PYZus{}list} \PY{o}{=}\PY{p}{[}\PY{p}{]}
          
          \PY{k}{for} \PY{n}{train\PYZus{}index}\PY{p}{,} \PY{n}{test\PYZus{}index} \PY{o+ow}{in} \PY{n}{skf}\PY{o}{.}\PY{n}{split}\PY{p}{(}\PY{n}{X}\PY{p}{,} \PY{n}{y}\PY{p}{)}\PY{p}{:}
              \PY{n}{X\PYZus{}train}\PY{p}{,} \PY{n}{X\PYZus{}test} \PY{o}{=} \PY{n}{X}\PY{p}{[}\PY{n}{train\PYZus{}index}\PY{p}{]}\PY{p}{,} \PY{n}{X}\PY{p}{[}\PY{n}{test\PYZus{}index}\PY{p}{]}
              \PY{n}{y\PYZus{}train}\PY{p}{,} \PY{n}{y\PYZus{}test} \PY{o}{=} \PY{n}{y}\PY{p}{[}\PY{n}{train\PYZus{}index}\PY{p}{]}\PY{p}{,} \PY{n}{y}\PY{p}{[}\PY{n}{test\PYZus{}index}\PY{p}{]}
              \PY{n}{clf} \PY{o}{=} \PY{n}{tree}\PY{o}{.}\PY{n}{DecisionTreeRegressor}\PY{p}{(}\PY{p}{)}\PY{o}{.}\PY{n}{fit}\PY{p}{(}\PY{n}{X\PYZus{}train}\PY{p}{,} \PY{n}{y\PYZus{}train}\PY{p}{)}
              \PY{n}{mse} \PY{o}{=} \PY{n}{mean\PYZus{}squared\PYZus{}error}\PY{p}{(}\PY{n}{y\PYZus{}test}\PY{p}{,} \PY{n}{clf}\PY{o}{.}\PY{n}{predict}\PY{p}{(}\PY{n}{X\PYZus{}test}\PY{p}{)}\PY{p}{)}
              \PY{n}{tree\PYZus{}mse\PYZus{}list}\PY{o}{.}\PY{n}{append}\PY{p}{(}\PY{n}{mse}\PY{p}{)}
             
\end{Verbatim}


    \begin{Verbatim}[commandchars=\\\{\}]
{\color{incolor}In [{\color{incolor}160}]:} \PY{c+c1}{\PYZsh{}let\PYZsq{}s try a more advanced boosting model GradientBoostingRegressor}
          
          \PY{n}{GBR\PYZus{}mse\PYZus{}list}\PY{o}{=}\PY{p}{[}\PY{p}{]}
          \PY{k}{for} \PY{n}{train\PYZus{}index}\PY{p}{,} \PY{n}{test\PYZus{}index} \PY{o+ow}{in} \PY{n}{skf}\PY{o}{.}\PY{n}{split}\PY{p}{(}\PY{n}{X}\PY{p}{,} \PY{n}{y}\PY{p}{)}\PY{p}{:}
              \PY{n}{X\PYZus{}train}\PY{p}{,} \PY{n}{X\PYZus{}test} \PY{o}{=} \PY{n}{X}\PY{p}{[}\PY{n}{train\PYZus{}index}\PY{p}{]}\PY{p}{,} \PY{n}{X}\PY{p}{[}\PY{n}{test\PYZus{}index}\PY{p}{]}
              \PY{n}{y\PYZus{}train}\PY{p}{,} \PY{n}{y\PYZus{}test} \PY{o}{=} \PY{n}{y}\PY{p}{[}\PY{n}{train\PYZus{}index}\PY{p}{]}\PY{p}{,} \PY{n}{y}\PY{p}{[}\PY{n}{test\PYZus{}index}\PY{p}{]}
              \PY{n}{est} \PY{o}{=} \PY{n}{GradientBoostingRegressor}\PY{p}{(}\PY{n}{n\PYZus{}estimators}\PY{o}{=}\PY{l+m+mi}{100}\PY{p}{,} \PY{n}{learning\PYZus{}rate}\PY{o}{=}\PY{l+m+mf}{0.1}\PY{p}{,} 
              \PY{n}{max\PYZus{}depth}\PY{o}{=}\PY{l+m+mi}{1}\PY{p}{,} \PY{n}{random\PYZus{}state}\PY{o}{=}\PY{l+m+mi}{0}\PY{p}{,} \PY{n}{loss}\PY{o}{=}\PY{l+s+s1}{\PYZsq{}}\PY{l+s+s1}{ls}\PY{l+s+s1}{\PYZsq{}}\PY{p}{)}\PY{o}{.}\PY{n}{fit}\PY{p}{(}\PY{n}{X\PYZus{}train}\PY{p}{,} \PY{n}{y\PYZus{}train}\PY{p}{)}
              \PY{n}{pred} \PY{o}{=} \PY{n}{est}\PY{o}{.}\PY{n}{predict}\PY{p}{(}\PY{n}{X\PYZus{}test}\PY{p}{)}
              \PY{n}{mse} \PY{o}{=} \PY{n}{mean\PYZus{}squared\PYZus{}error}\PY{p}{(}\PY{n}{y\PYZus{}test}\PY{p}{,} \PY{n}{est}\PY{o}{.}\PY{n}{predict}\PY{p}{(}\PY{n}{X\PYZus{}test}\PY{p}{)}\PY{p}{)}
              \PY{n}{GBR\PYZus{}mse\PYZus{}list}\PY{o}{.}\PY{n}{append}\PY{p}{(}\PY{n}{mse}\PY{p}{)}
              
\end{Verbatim}


    \begin{Verbatim}[commandchars=\\\{\}]
{\color{incolor}In [{\color{incolor}163}]:} \PY{c+c1}{\PYZsh{} XGBoost with 10\PYZhy{}fold\PYZhy{} stratified cross validation}
          \PY{n}{XGB\PYZus{}mse\PYZus{}list}\PY{o}{=}\PY{p}{[}\PY{p}{]}
          \PY{k}{for} \PY{n}{train\PYZus{}index}\PY{p}{,} \PY{n}{test\PYZus{}index} \PY{o+ow}{in} \PY{n}{skf}\PY{o}{.}\PY{n}{split}\PY{p}{(}\PY{n}{X}\PY{p}{,} \PY{n}{y}\PY{p}{)}\PY{p}{:}
              \PY{n}{X\PYZus{}train}\PY{p}{,} \PY{n}{X\PYZus{}test} \PY{o}{=} \PY{n}{X}\PY{p}{[}\PY{n}{train\PYZus{}index}\PY{p}{]}\PY{p}{,} \PY{n}{X}\PY{p}{[}\PY{n}{test\PYZus{}index}\PY{p}{]}
              \PY{n}{y\PYZus{}train}\PY{p}{,} \PY{n}{y\PYZus{}test} \PY{o}{=} \PY{n}{y}\PY{p}{[}\PY{n}{train\PYZus{}index}\PY{p}{]}\PY{p}{,} \PY{n}{y}\PY{p}{[}\PY{n}{test\PYZus{}index}\PY{p}{]}
              \PY{n}{xgb} \PY{o}{=} \PY{n}{XGBRegressor}\PY{p}{(}\PY{n}{n\PYZus{}estimators}\PY{o}{=}\PY{l+m+mi}{500}\PY{p}{,} \PY{n}{learning\PYZus{}rate}\PY{o}{=}\PY{l+m+mf}{0.05}\PY{p}{,} \PY{n}{subsample}\PY{o}{=}\PY{l+m+mf}{0.5}\PY{p}{,} \PY{n}{colsample\PYZus{}bytree}\PY{o}{=}\PY{l+m+mf}{0.5}\PY{p}{,} 
                             \PY{n}{max\PYZus{}depth}\PY{o}{=}\PY{l+m+mi}{3}\PY{p}{,} \PY{n}{gamma}\PY{o}{=}\PY{l+m+mi}{0}\PY{p}{,} \PY{n}{reg\PYZus{}alpha}\PY{o}{=}\PY{l+m+mi}{0}\PY{p}{,} \PY{n}{reg\PYZus{}lambda}\PY{o}{=}\PY{l+m+mi}{2}\PY{p}{,} \PY{n}{min\PYZus{}child\PYZus{}weight}\PY{o}{=}\PY{l+m+mi}{1}\PY{p}{)}\PY{o}{.}\PY{n}{fit}\PY{p}{(}\PY{n}{X\PYZus{}train}\PY{p}{,} \PY{n}{y\PYZus{}train}\PY{p}{)}
              \PY{n}{pred} \PY{o}{=} \PY{n}{xgb}\PY{o}{.}\PY{n}{predict}\PY{p}{(}\PY{n}{X\PYZus{}test}\PY{p}{)}
              \PY{n}{mse} \PY{o}{=} \PY{n}{mean\PYZus{}squared\PYZus{}error}\PY{p}{(}\PY{n}{y\PYZus{}test}\PY{p}{,} \PY{n}{xgb}\PY{o}{.}\PY{n}{predict}\PY{p}{(}\PY{n}{X\PYZus{}test}\PY{p}{)}\PY{p}{)}
              \PY{n}{XGB\PYZus{}mse\PYZus{}list}\PY{o}{.}\PY{n}{append}\PY{p}{(}\PY{n}{mse}\PY{p}{)}
\end{Verbatim}


    \subsection{Summary of three regression
models}\label{summary-of-three-regression-models}

    \begin{Verbatim}[commandchars=\\\{\}]
{\color{incolor}In [{\color{incolor}165}]:} \PY{k}{print} \PY{l+s+s1}{\PYZsq{}}\PY{l+s+s1}{The Cross\PYZus{}validation LOSS of decision tree regressor is}\PY{l+s+s1}{\PYZsq{}}
          \PY{k}{print} \PY{n}{np}\PY{o}{.}\PY{n}{mean}\PY{p}{(}\PY{n}{tree\PYZus{}mse\PYZus{}list}\PY{p}{)}
          \PY{k}{print} \PY{l+s+s1}{\PYZsq{}}\PY{l+s+s1}{The Cross\PYZus{}validation LOSS of GradientBoosting Regressor is}\PY{l+s+s1}{\PYZsq{}}
          \PY{k}{print} \PY{n}{np}\PY{o}{.}\PY{n}{mean}\PY{p}{(}\PY{n}{GBR\PYZus{}mse\PYZus{}list}\PY{p}{)}
          \PY{k}{print} \PY{l+s+s1}{\PYZsq{}}\PY{l+s+s1}{The Cross\PYZus{}validation LOSS of XGBRegressor is}\PY{l+s+s1}{\PYZsq{}}
          \PY{k}{print} \PY{n}{np}\PY{o}{.}\PY{n}{mean}\PY{p}{(}\PY{n}{XGB\PYZus{}mse\PYZus{}list}\PY{p}{)}
\end{Verbatim}


    \begin{Verbatim}[commandchars=\\\{\}]
The Cross\_validation LOSS of decision tree regressor is
0.000457000270037
The Cross\_validation LOSS of GradientBoosting Regressor is
0.000243897601461
The Cross\_validation LOSS of XGBRegressor is
0.000243081547549

    \end{Verbatim}

    \subsubsection{As the loss of XGBRegressor is the lowest according to
cross-validation results. Therefore, we will choose XGBRegressor among
the three
models}\label{as-the-loss-of-xgbregressor-is-the-lowest-according-to-cross-validation-results.-therefore-we-will-choose-xgbregressor-among-the-three-models}

    \section{Step 4: Hyperparameter
tuning}\label{step-4-hyperparameter-tuning}

\begin{enumerate}
\def\labelenumi{\arabic{enumi}.}
\item
  The hyperparameters of these regressors were tuned using GridSearchCV.
\item
  Again, becasue of my computer computational ability is not good, it is
  very slow to run GridSearchCV over many different options of
  parameters. Therefore, I did not try different parameters a lot, but
  we can get the idea of hyperparameter tuning
\end{enumerate}

    \begin{Verbatim}[commandchars=\\\{\}]
{\color{incolor}In [{\color{incolor}181}]:} \PY{n}{params}\PY{o}{=}\PY{p}{\PYZob{}}
              \PY{l+s+s1}{\PYZsq{}}\PY{l+s+s1}{max\PYZus{}depth}\PY{l+s+s1}{\PYZsq{}}\PY{p}{:} \PY{p}{[}\PY{l+m+mi}{3}\PY{p}{,}\PY{l+m+mi}{5}\PY{p}{]}\PY{p}{,} \PY{c+c1}{\PYZsh{} 5 is good but takes too long in kaggle env}
              \PY{c+c1}{\PYZsh{}\PYZsq{}subsample\PYZsq{}: [0.3,0.5],}
              \PY{c+c1}{\PYZsh{}\PYZsq{}colsample\PYZus{}bytree\PYZsq{}: [0.5,0.8],}
              \PY{l+s+s1}{\PYZsq{}}\PY{l+s+s1}{n\PYZus{}estimators}\PY{l+s+s1}{\PYZsq{}}\PY{p}{:} \PY{p}{[}\PY{l+m+mi}{500}\PY{p}{,}\PY{l+m+mi}{100}\PY{p}{]}\PY{p}{,}
              \PY{c+c1}{\PYZsh{}\PYZsq{}reg\PYZus{}alpha\PYZsq{}: [0.01, 0]}
          \PY{p}{\PYZcb{}}
          \PY{n}{xgb\PYZus{}clf} \PY{o}{=} \PY{n}{XGBRegressor}\PY{p}{(}\PY{p}{)}
          \PY{n}{rs} \PY{o}{=} \PY{n}{GridSearchCV}\PY{p}{(}\PY{n}{xgb\PYZus{}clf}\PY{p}{,}
                            \PY{n}{params}\PY{p}{,}
                            \PY{n}{cv}\PY{o}{=}\PY{l+m+mi}{5}\PY{p}{,}
                            \PY{n}{scoring}\PY{o}{=}\PY{l+s+s2}{\PYZdq{}}\PY{l+s+s2}{neg\PYZus{}mean\PYZus{}squared\PYZus{}error}\PY{l+s+s2}{\PYZdq{}}\PY{p}{,}
                            \PY{n}{n\PYZus{}jobs}\PY{o}{=}\PY{l+m+mi}{1}\PY{p}{,}
                            \PY{n}{verbose}\PY{o}{=}\PY{l+m+mi}{2}\PY{p}{)}
          \PY{n}{rs}\PY{o}{.}\PY{n}{fit}\PY{p}{(}\PY{n}{X\PYZus{}train}\PY{p}{,} \PY{n}{y\PYZus{}train}\PY{p}{)}
          \PY{n}{best\PYZus{}est} \PY{o}{=} \PY{n}{rs}\PY{o}{.}\PY{n}{best\PYZus{}estimator\PYZus{}}
          \PY{k}{print}\PY{p}{(}\PY{n}{best\PYZus{}est}\PY{p}{)}
\end{Verbatim}


    \begin{Verbatim}[commandchars=\\\{\}]
Fitting 5 folds for each of 4 candidates, totalling 20 fits
[CV] n\_estimators=500, max\_depth=3 {\ldots}
[CV] {\ldots} n\_estimators=500, max\_depth=3, total= 1.7min
[CV] n\_estimators=500, max\_depth=3 {\ldots}

    \end{Verbatim}

    \begin{Verbatim}[commandchars=\\\{\}]
[Parallel(n\_jobs=1)]: Done   1 out of   1 | elapsed:  1.8min remaining:    0.0s

    \end{Verbatim}

    \begin{Verbatim}[commandchars=\\\{\}]
[CV] {\ldots} n\_estimators=500, max\_depth=3, total= 1.7min
[CV] n\_estimators=500, max\_depth=3 {\ldots}
[CV] {\ldots} n\_estimators=500, max\_depth=3, total= 1.7min
[CV] n\_estimators=500, max\_depth=3 {\ldots}
[CV] {\ldots} n\_estimators=500, max\_depth=3, total= 1.6min
[CV] n\_estimators=500, max\_depth=3 {\ldots}
[CV] {\ldots} n\_estimators=500, max\_depth=3, total= 1.6min
[CV] n\_estimators=100, max\_depth=3 {\ldots}
[CV] {\ldots} n\_estimators=100, max\_depth=3, total=  19.3s
[CV] n\_estimators=100, max\_depth=3 {\ldots}
[CV] {\ldots} n\_estimators=100, max\_depth=3, total=  19.2s
[CV] n\_estimators=100, max\_depth=3 {\ldots}
[CV] {\ldots} n\_estimators=100, max\_depth=3, total=  19.3s
[CV] n\_estimators=100, max\_depth=3 {\ldots}
[CV] {\ldots} n\_estimators=100, max\_depth=3, total=  19.4s
[CV] n\_estimators=100, max\_depth=3 {\ldots}
[CV] {\ldots} n\_estimators=100, max\_depth=3, total=  20.3s
[CV] n\_estimators=500, max\_depth=5 {\ldots}
[CV] {\ldots} n\_estimators=500, max\_depth=5, total= 2.9min
[CV] n\_estimators=500, max\_depth=5 {\ldots}
[CV] {\ldots} n\_estimators=500, max\_depth=5, total= 2.8min
[CV] n\_estimators=500, max\_depth=5 {\ldots}
[CV] {\ldots} n\_estimators=500, max\_depth=5, total= 2.8min
[CV] n\_estimators=500, max\_depth=5 {\ldots}
[CV] {\ldots} n\_estimators=500, max\_depth=5, total= 3.0min
[CV] n\_estimators=500, max\_depth=5 {\ldots}
[CV] {\ldots} n\_estimators=500, max\_depth=5, total= 2.8min
[CV] n\_estimators=100, max\_depth=5 {\ldots}
[CV] {\ldots} n\_estimators=100, max\_depth=5, total=  32.2s
[CV] n\_estimators=100, max\_depth=5 {\ldots}
[CV] {\ldots} n\_estimators=100, max\_depth=5, total=  32.0s
[CV] n\_estimators=100, max\_depth=5 {\ldots}
[CV] {\ldots} n\_estimators=100, max\_depth=5, total=  31.2s
[CV] n\_estimators=100, max\_depth=5 {\ldots}
[CV] {\ldots} n\_estimators=100, max\_depth=5, total=  30.9s
[CV] n\_estimators=100, max\_depth=5 {\ldots}
[CV] {\ldots} n\_estimators=100, max\_depth=5, total=  30.9s

    \end{Verbatim}

    \begin{Verbatim}[commandchars=\\\{\}]
[Parallel(n\_jobs=1)]: Done  20 out of  20 | elapsed: 28.4min finished

    \end{Verbatim}

    \begin{Verbatim}[commandchars=\\\{\}]
XGBRegressor(base\_score=0.5, booster='gbtree', colsample\_bylevel=1,
       colsample\_bytree=1, gamma=0, learning\_rate=0.1, max\_delta\_step=0,
       max\_depth=3, min\_child\_weight=1, missing=None, n\_estimators=100,
       n\_jobs=1, nthread=None, objective='reg:linear', random\_state=0,
       reg\_alpha=0, reg\_lambda=1, scale\_pos\_weight=1, seed=None,
       silent=True, subsample=1)

    \end{Verbatim}

    \subsubsection{After finding the best parameters, Now we can use these
parameter in the XGBoosting regression
model}\label{after-finding-the-best-parameters-now-we-can-use-these-parameter-in-the-xgboosting-regression-model}

next, we will use this model to make the final prediction on test set
and computet the losses for each records in the test set and save these
prediction into a csv file

\subsubsection{Finally, as we already have the labels for the test set,
we can compute the L2
loss}\label{finally-as-we-already-have-the-labels-for-the-test-set-we-can-compute-the-l2-loss}

    \begin{Verbatim}[commandchars=\\\{\}]
{\color{incolor}In [{\color{incolor}236}]:} \PY{c+c1}{\PYZsh{} change several paramters as the best options and the we will use this as the final model}
          
          \PY{n}{xgb} \PY{o}{=} \PY{n}{XGBRegressor}\PY{p}{(}\PY{n}{base\PYZus{}score}\PY{o}{=}\PY{l+m+mf}{0.5}\PY{p}{,} \PY{n}{booster}\PY{o}{=}\PY{l+s+s1}{\PYZsq{}}\PY{l+s+s1}{gbtree}\PY{l+s+s1}{\PYZsq{}}\PY{p}{,} \PY{n}{colsample\PYZus{}bylevel}\PY{o}{=}\PY{l+m+mi}{1}\PY{p}{,}
                 \PY{n}{colsample\PYZus{}bytree}\PY{o}{=}\PY{l+m+mi}{1}\PY{p}{,} \PY{n}{gamma}\PY{o}{=}\PY{l+m+mi}{0}\PY{p}{,} \PY{n}{learning\PYZus{}rate}\PY{o}{=}\PY{l+m+mf}{0.1}\PY{p}{,} \PY{n}{max\PYZus{}delta\PYZus{}step}\PY{o}{=}\PY{l+m+mi}{0}\PY{p}{,}
                 \PY{n}{max\PYZus{}depth}\PY{o}{=}\PY{l+m+mi}{3}\PY{p}{,} \PY{n}{min\PYZus{}child\PYZus{}weight}\PY{o}{=}\PY{l+m+mi}{1}\PY{p}{,} \PY{n}{missing}\PY{o}{=}\PY{n+nb+bp}{None}\PY{p}{,} \PY{n}{n\PYZus{}estimators}\PY{o}{=}\PY{l+m+mi}{100}\PY{p}{,}
                 \PY{n}{n\PYZus{}jobs}\PY{o}{=}\PY{l+m+mi}{1}\PY{p}{,} \PY{n}{nthread}\PY{o}{=}\PY{n+nb+bp}{None}\PY{p}{,} \PY{n}{objective}\PY{o}{=}\PY{l+s+s1}{\PYZsq{}}\PY{l+s+s1}{reg:linear}\PY{l+s+s1}{\PYZsq{}}\PY{p}{,} \PY{n}{random\PYZus{}state}\PY{o}{=}\PY{l+m+mi}{0}\PY{p}{,}
                 \PY{n}{reg\PYZus{}alpha}\PY{o}{=}\PY{l+m+mi}{0}\PY{p}{,} \PY{n}{reg\PYZus{}lambda}\PY{o}{=}\PY{l+m+mi}{1}\PY{p}{,} \PY{n}{scale\PYZus{}pos\PYZus{}weight}\PY{o}{=}\PY{l+m+mi}{1}\PY{p}{,} \PY{n}{seed}\PY{o}{=}\PY{n+nb+bp}{None}\PY{p}{,}
                 \PY{n}{silent}\PY{o}{=}\PY{n+nb+bp}{True}\PY{p}{,} \PY{n}{subsample}\PY{o}{=}\PY{l+m+mi}{1}\PY{p}{)}\PY{o}{.}\PY{n}{fit}\PY{p}{(}\PY{n}{X\PYZus{}train}\PY{p}{,} \PY{n}{y\PYZus{}train}\PY{p}{)}
\end{Verbatim}


    \section{When we have some data need to predict the probabiity of click,
we can use this
mode}\label{when-we-have-some-data-need-to-predict-the-probabiity-of-click-we-can-use-this-mode}

\paragraph{Note that the following code is for the new data that need to
perform prediction; the following code won't
work}\label{note-that-the-following-code-is-for-the-new-data-that-need-to-perform-prediction-the-following-code-wont-work}

\paragraph{The new data need to go through the data preproceiing steps,
then we can use this model to predict the probability of
click}\label{the-new-data-need-to-go-through-the-data-preproceiing-steps-then-we-can-use-this-model-to-predict-the-probability-of-click}

    \begin{Verbatim}[commandchars=\\\{\}]
{\color{incolor}In [{\color{incolor} }]:} \PY{n}{X\PYZus{}pred} \PY{o}{=} \PY{n}{pd}\PY{o}{.}\PY{n}{read\PYZus{}csv}\PY{p}{(}\PY{l+s+s1}{\PYZsq{}}\PY{l+s+s1}{./ ....}\PY{l+s+s1}{\PYZsq{}}\PY{p}{)}
        \PY{n}{pred} \PY{o}{=} \PY{n}{xgb}\PY{o}{.}\PY{n}{predict}\PY{p}{(}\PY{n}{X\PYZus{}pred}\PY{p}{)}
        \PY{n}{mse} \PY{o}{=} \PY{n}{mean\PYZus{}squared\PYZus{}error}\PY{p}{(}\PY{n}{y\PYZus{}test}\PY{p}{,} \PY{n}{xgb}\PY{o}{.}\PY{n}{predict}\PY{p}{(}\PY{n}{X\PYZus{}test}\PY{p}{)}\PY{p}{)}
\end{Verbatim}


    \begin{Verbatim}[commandchars=\\\{\}]
{\color{incolor}In [{\color{incolor}233}]:} \PY{n}{cilck\PYZus{}probs} \PY{o}{=} \PY{n}{pred}
          
          \PY{c+c1}{\PYZsh{}generate the dataFrame we want to save.}
          \PY{n}{pred}\PY{o}{=} \PY{n}{np}\PY{o}{.}\PY{n}{reshape}\PY{p}{(}\PY{n}{pred}\PY{p}{,} \PY{p}{(}\PY{n}{pred}\PY{o}{.}\PY{n}{shape}\PY{p}{[}\PY{l+m+mi}{0}\PY{p}{]}\PY{p}{,} \PY{l+m+mi}{1}\PY{p}{)}\PY{p}{)} 
          \PY{n}{test\PYZus{}arr} \PY{o}{=} \PY{n}{np}\PY{o}{.}\PY{n}{hstack}\PY{p}{(}\PY{p}{(}\PY{n}{X\PYZus{}test}\PY{p}{,}\PY{n}{y\PYZus{}test}\PY{p}{,}\PY{n}{pred}\PY{p}{)}\PY{p}{)}
          \PY{n}{col} \PY{o}{=}\PY{n}{raw\PYZus{}data}\PY{o}{.}\PY{n}{columns}\PY{o}{.}\PY{n}{tolist}\PY{p}{(}\PY{p}{)}
          \PY{n}{colu} \PY{o}{=}\PY{n}{col}\PY{o}{+}\PY{p}{[}\PY{l+s+s1}{\PYZsq{}}\PY{l+s+s1}{Prob\PYZus{}Click}\PY{l+s+s1}{\PYZsq{}}\PY{p}{]}
          \PY{n}{df\PYZus{}re}\PY{o}{=} \PY{n}{pd}\PY{o}{.}\PY{n}{DataFrame}\PY{p}{(}\PY{n}{test\PYZus{}arr}\PY{p}{,}\PY{n}{columns}\PY{o}{=}\PY{n}{colu}\PY{p}{)}
          \PY{c+c1}{\PYZsh{}\PYZsh{} Save the prediction results to a .csv file as dataframe}
          \PY{n}{save\PYZus{}results} \PY{o}{=} \PY{n}{df\PYZus{}re}
          \PY{n}{save\PYZus{}results}\PY{o}{.}\PY{n}{to\PYZus{}csv}\PY{p}{(}\PY{l+s+s2}{\PYZdq{}}\PY{l+s+s2}{xgboost\PYZus{}Click\PYZus{}Probability.csv}\PY{l+s+s2}{\PYZdq{}}\PY{p}{,} \PY{n}{index}\PY{o}{=}\PY{n+nb+bp}{False}\PY{p}{)}
\end{Verbatim}


    \section{Summary}\label{summary}

\subsection{In order to predict the view-to-click probability of online
ads, the following steps are
taken:}\label{in-order-to-predict-the-view-to-click-probability-of-online-ads-the-following-steps-are-taken}

\subsubsection{1. Data analysis and data
wrangling}\label{data-analysis-and-data-wrangling}

\subsubsection{2. Feature Selection and Feature
Engineering}\label{feature-selection-and-feature-engineering}

\subsubsection{3. Model Selection and Model
building}\label{model-selection-and-model-building}

\subsubsection{4. Model Evaluation and Hyperparameter
Tuning}\label{model-evaluation-and-hyperparameter-tuning}

    Thanks for giving me this opportunity, any feedback will be appreciate.


    % Add a bibliography block to the postdoc
    
    
    
    \end{document}
